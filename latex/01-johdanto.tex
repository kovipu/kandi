\chapter{Johdanto} \label{Johdanto}
Tietotekniikan käyttö arkielämässä on yleistynyt viime vuosina räjähdysmäisesti. Etenkin verkkosivujen ja -sovellusten
käyttö on suositumpaa kuin koskaan. Kasvanut käyttö on lisännyt merkittäväst web-tekniikoille asetettuja vaatimuksia
muun muassa luotettavuuden ja produktiivisuuden kannalta. 

Funktionaalinen ohjelmointi on matemaattisten funktioiden evaluointiin perustuva ohjelmointiparadigma. Funktionaalinen
ohjelmointi ei ole paradigmana uusi keksintö, mutta sen soveltaminen web-kehitykseen on nostanut päätään vasta viime
vuosina. Tämän tutkielman tavoitteena on selvittää, miten funktionaalinen ohjelmointi soveltuu frontend-web-kehitykseen.

Tutkielma käsittelee funktionaalista ohjelmointia yleisellä tasolla ja lisäksi tutkii kahden modernin
funktionaalisreaktiivisen frontend-tekniikan toteutusta funktionaalisuudelle. Tutkittaviksi teknologioiksi on valittu
React-kirjasto ja Elm-ohjelmointikieli, jotka molemmat kääntyvät selaimen tulkittavaksi JavaScript-kieleksi.

Funktionaalista ohjelmoinnin konsepteja ja syntyä käsittelee tutkielman luku 2. Luvussa selvitetään myös hyötyjä
funktionaalisen ohjelmoinnin käytöstä.

Funktionaalista ohjelmointia web-kehityksessä ja frontend-sovelluksen toteutuksessa etenkin funktionaalisen reaktiivisen
ohjelmoinnin avulla käsittelevät luvut 3 ja 4.
