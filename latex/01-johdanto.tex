\chapter{Johdanto} \label{Johdanto}
Tietotekniikan käyttö arkielämässä on yleistynyt viime vuosina räjähdysmäisesti. Etenkin verkkosivujen ja -sovellusten
käyttö on suositumpaa kuin koskaan. Kasvanut käyttö on lisännyt merkittäväst web-tekniikoille asetettuja vaatimuksia
muun muassa luotettavuuden ja tuottavuuden kannalta. 

Funktionaalinen ohjelmointi on matemaattisten funktioiden evaluointiin perustuva ohjelmointiparadigma. Funktionaalinen
ohjelmointi ei ole paradigmana uusi keksintö, mutta sen soveltaminen web-kehitykseen on nostanut päätään vasta viime
vuosina. Tämän tutkielman tavoitteena on selvittää, miten funktionaalinen ohjelmointi soveltuu frontend-web-kehitykseen.

Tutkielma käsittelee funktionaalista ohjelmointia yleisellä tasolla ja lisäksi tutkii kahta modernia funktionaalista
reaktiivista frontend-teknologiaa. Tutkittaviksi teknologioiksi on valittu
React-kirjasto\footnote{React-kirjaston verkkosivusto \url{https://https://reactjs.org/}.} ja
Elm-ohjelmointikieli\footnote{Elm-ohjelmointikielen verkkosivusto \url{https://elm-lang.org/}.}, jotka molemmat
kääntyvät selaimen tulkittavaksi JavaScript-kieleksi.

Tutkielman luku 2 käsitteleen funktionaalisen ohjelmoinnin syntyä ja konsepteja. Luvussa selvitetään myös hyötyjä
funktionaalisen ohjelmoinnin käytöstä.

Tutkielman luvut 3 ja 4 käsittelevät funktionaalista ohjelmointia web-kehitykses\-sä ja frontend-sovelluksen toteutuksessa
etenkin funktionaalisen reaktiivisen ohjelmoinnin avulla.
