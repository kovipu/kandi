\documentclass[a4paper,12pt,language=finnish,version=draft,hidechapters=true,includereferences=false,realtimesnewroman=false,sharelatex=false,emptyfirstpages=true]{utuftthesis}
\setcounter{secnumdepth}{2}
\setcounter{tocdepth}{2}

\addbibresource{Bibliografia.bib}
\usepackage{minted}
\begin{document}
\begin{comment}
Document template suitable for use as a LaTeX master-file for master's
thesis in University of Turku Department of Future Technologies.\\
\\
Compatible with: ShareLaTeX / PDFLaTeX / XeLaTeX.\\
\\
\\
{*}{*} HOW TO USE? {*}{*}\\
\\
Want to write a thesis? Clone this template in ShareLaTeX or fork
the department's thesis git project.\\
\\
The utuftthesis.cls defines a new thesis class, which is based on
the report class. It supports these new named parameters:

- paper: a4paper

- version: draft / final (default: draft) shows/hides {[}draft{]}
in the header

- language: finnish / english (default: finnish) affects the general
document appearance and hyphenation

- hidechapters: true / false (default: true) hides/shows the chapter/luku
text at the beginning of each Chapter

- includereferences: true / false (default: false) include reference
pages when calculating the total number of pages

- realtimesnewroman: true / false (default: false) use Times New Roman
instead of LaTeX fonts with XeLaTeX. Requires the font to be installed
on the system / provided in the document directory. Other fonts can
be defined with \textbackslash setmainfont.

- sharelatex: true / false (default: false) don't attempt to use (c)
system fonts, instead read them from the project repository

- emptyfirstpages: true / false (default: true) clear the headers/footers
for the 1st pages of text chapters

Traditionally the best places to learn (La)TeX are probably the manual
pages for each package http://www.ctan.org/ and http://www.ctan.org/tex-archive/info/lshort/english/lshort.pdf.
This new version (2.0) should be compatible with xelatex and biblatex
which means that all source files can freely use normal UTF-8 text
without resorting to \textquotedbl\textquotedbl legacy hacks\textquotedbl .\\
\\
Note that PDF/A requirements don't allow PDF links, but if you want
to provide a user friendly version of the thesis with links, use \textbackslash hyperref\\
\\
\\
{*}{*} Maintenance {*}{*}\\
\\
Workflow: https://gitlab.utu.fi/ttweb/thesis -> master .lyx document
exported as .tex documents -> repository content dumped to the sharelatex
project template

Want to fix something in the template? Send a merge.\\
\\
Relies on utuftthesis.cls for the document class definitions.
\end{comment}


\pubyear{2019}

\pubmonth{6}

\publaben{Software Engineering}
\publab{Ohjelmistotekniikka}

\pubtype{tkk}
\title{Funktionaalinen ohjelmointi front-end web-kehityksessä}
\author{Konsta Purtsi}

\maketitle
\keywords{funktionaalinen ohjelmointi, funktionaalinen reaktiivinen ohjelmointi, web-ohjelmointi, frontend, React, Elm}
\keywordstwo{functional programming, functional reactive programming, web-programming, frontend, React, Elm}

\begin{abstract}
Tietotekniikan käyttö arkielämässä on yleistynyt viime vuosina räjähdysmäisesti. Etenkin verkkosivujen ja -sovellusten
käyttö on suositumpaa kuin koskaan. Kasvanut käyttö on lisännyt merkittäväst web-tekniikoille asetettuja vaatimuksia
muun muassa luotettavuuden ja tuottavuuden kannalta. 

Funktionaalinen ohjelmointi on matemaattisten funktioiden evaluointiin perustuva ohjelmointiparadigma. Funktionaalinen
ohjelmointi ei ole paradigmana uusi keksintö, mutta sen soveltaminen web-kehitykseen on nostanut päätään vasta viime
vuosina. Tämän tutkielman tavoitteena on selvittää, miten funktionaalinen ohjelmointi soveltuu frontend-web-kehitykseen.

Tutkielmassa selvitetään ensin funktionaalisen ohjelmoinnin konsepteja ja etuja. Tämän jälkeen tarkastellaan
web-kehitystä, ja miten funktionaalinen ohjelmointi soveltuu frontend-web-sovelluksen toteutukseen. Tutkielmassa
tarkastellaan tarkemmin kahta funktionaalista reaktiivista frontend-ohjelmointitekniikkaa: React-kirjastoa ja
Elm-ohjelmointikieltä.

Funktionaalinen ohjelmointi ja etenkin funktionaalinen reaktiivinen ohjelmointi on hyvin luonteva tapa toteuttaa
frontend web-sovellus. Tutkituista tekniikoista molemmat tukevat funktionaalista ohjelmointityyliä, mutta Elm ei
myöskään salli muuta tapaa toteuttaa ohjelma.
\end{abstract}


% empty pagestyle for table of contents etc.
% otherwise you'll get simple page style with roman page numbers
\pagestyle{empty}

% mandatory
\tableofcontents

% if you want a list of figures
%\listoffigures

% if you want a list of tables
%\listoftables

% 'list of acronyms'
%   - you may not need this at all
%   - create a chapter called List Of Acronyms (or whatever), which
%     should contain all your acronym definitions, e.g.
%     \chapter{List Of Acronyms}
%   - the secnumdepth trickery is needed because acronyms are as a
%     standard chapter and we are faking '\listofacronyms'
%
%\setcounter{secnumdepth}{-1}
%\input{your acronym chapter's file name}
%\setcounter{secnumdepth}{2}% setup page numbering, page counter, etc.%
\begin{comment}
The thesis starts here.

To better organize things, create a new tex file for each chapter
and input it below.

Avoid using the å, ä, ö or <space> characters in referred names and
underscores \_ in file names (may break hyperref).

Good luck!
\end{comment}

\chapter{Johdanto} \label{Johdanto}
Tietotekniikan käyttö arkielämässä on yleistynyt viime vuosina räjähdysmäisesti. Etenkin verkkosivujen ja -sovellusten
käyttö on suositumpaa kuin koskaan. Kasvanut käyttö on lisännyt merkittäväst web-tekniikoille asetettuja vaatimuksia
muun muassa luotettavuuden ja produktiivisuuden kannalta. 

Funktionaalinen ohjelmointi on matemaattisten funktioiden evaluointiin perustuva ohjelmointiparadigma. Funktionaalinen
ohjelmointi ei ole paradigmana uusi keksintö, mutta sen soveltaminen web-kehitykseen on nostanut päätään vasta viime
vuosina. Tämän tutkielman tavoitteena on selvittää, miten funktionaalinen ohjelmointi soveltuu frontend-web-kehitykseen.

Tutkielma käsittelee funktionaalista ohjelmointia yleisellä tasolla ja lisäksi tutkii kahden modernin
funktionaalisreaktiivisen frontend-tekniikan toteutusta funktionaalisuudelle. Tutkittaviksi teknologioiksi on valittu
React-kirjasto ja Elm-ohjelmointikieli, jotka molemmat kääntyvät selaimen tulkittavaksi JavaScript-kieleksi.

Funktionaalista ohjelmoinnin konsepteja ja syntyä käsittelee tutkielman luku 2. Luvussa selvitetään myös hyötyjä
funktionaalisen ohjelmoinnin käytöstä.

Funktionaalista ohjelmointia web-kehityksessä ja frontend-sovelluksen toteutuksessa etenkin funktionaalisen reaktiivisen
ohjelmoinnin avulla käsittelevät luvut 3 ja 4.

\chapter{Funktionaalinen ohjelmointi} \label{Funktionaalinen ohjelmointi}

\section{Määritelmä}
Funktionaalinen ohjelmointi on deklaratiivinen ohjelmointiparadigma, jossa ohjelman suoritus tapahtuu sisäkkäisten
matemaattisten funktioiden evaluoimisena. Deklaratiivinen ohjelmointi on ohjelmointiparadigma, jossa kuvaillaan ohjelman
lopputulos tai tila. Deklaratiivisessa ohjelmoinnissa algoritmi kuvataan varsinaisen kontrollivuon sijaan globaalia
tilaa muokkaavin funktiokutsuin. Se on vastakohta perinteisemmälle imperatiiviselle ohjelmoinnille. Funktionaalinen
ohjelmakoodi pyrkii käsittelemään tilaa tarkemmin ja välttämään muuttuvia arvoja.\cite{hudak}

Puhtaasti funktionaalinen ohjelmointikieli ei salli lainkaan muuttuvaa tilaa tai datan muokkaamista. Tämä vaatii sen,
että funktioiden paluuarvoo riippuu vain ja ainoastaan funktion saamista parametreista, eikä mikään tila voi vaikuttaa
funktion paluuarvoon. Simppelien loogisten toimintojen lisäksi tosielämän ohjelmointikielet vaativat kuitenkin joitain
epäpuhtauksia ollakseen oikeasti hyödyllisiä.\cite{purelyFunctional}

\section{Lambdakalkyyli}
Funktionaalinen ohjelmointi perustuu vahvasti lambdakalkyyliin - formaalin laskennan malliin jonka kehitti Alonzo Church
vuonna 1928. Lambdakalkyyliä pidetään ensimmäisenä funktionaalisena ohjelmointikielenä, vaikka se keksittiinkin ennen
ensimmäisiä varsinaisia tietokoneita, eikä sitä pidettykään aikanaan ohjelmointikielenä. Useita moderneja
ohjelmointikieliä, kuten Lispiä, pidetään abstraktioina lambdakalkyylin päälle. Lambdakalkyyli voidaan kirjoittaa myös
$ \lambda $-kalkyyli kreikkalaisella lambda-kirjaimella.\cite{hudak}

Yksinkertaisimmillaan lambdakalkyyli koostuu vain kolmesta termistä: muuttujista, abstraktioista ja sovelluksista.
Muuttujat ovat yksinkertaisesti merkkejä tai merkkijonoja jotka kuvaavat jotain parametriä tai arvoa. Churchin
alkuperäinen lambdakalkyyli ei tuntenut muuttujien asettamista, ainoastaan arvojen syöttämisen parametrina. Hänen
lambdakalkyylinsä primitiivitermit olivat abstraktio ja sovellus:

\begin{center}
  \begin{tabular}{ |c|c|c| } 
    \hline
    Syntaksi & Nimi & Kuvaus \\ [0.5ex] 
    \hline
    $ \lambda x [ M ] $ & Abstraktio & Määrittelee funktion parametrille x toteutuksella M \\ 
    \hline
    $ \{ F \} ( X ) $ & Sovellus & suorittaa funktion F arvolla X \\ 
    \hline
  \end{tabular}
\end{center}\cite{lambdacalculus}

% - moderni lambdakalkyyli

\section{Konseptit}
% * declarative?

\subsection{Tilattomuus}
Deklaratiivisissa ohjelmointikielissä ei ole implisiittistä tilaa, mikä eroaa imperatiivisista kielistä, joissa tilaa
voi muokata lauseilla (eng. commands) ohjelmakoodissa. Tilattomuus tekee laskennasta mahdollista ainoastaan lausekkeilla
(eng. expression). Funktionaaliset ohjelmointikielet perustuvat lambdakalkyyliin ja siten käyttävät ja
lambda-abstraktioita kaiken laskennan perustana. Esimerkiksi logiikkaaohjelmoinnissa vastaava rakenne on relaatio.
Tilattomuus sallii vain muuttumattomat vakiot, mikä estää muuttujien arvon muuttamisen ja perinteiset silmukat.
Puhtaasti funktionaalisessa ohjelmoinnissa rekursio on ainoa tapa toteuttaa toisto. Esimerkiksi funktio, joka laskee
luvun \textit{n} kertoman voitaisiin toteuttaa imperatiivisesti Python-kielessä esimerkiksi näin:
\begin{minted}{python}
def factorial(n):
    result = 1
    while n >= 1:
        result = result * n
        n = n - 1
    return result
\end{minted}
Funktionaalisessa ohjelmointikielessä toistorakenne on toteutettava rekursion avulla. Funktionaaliset ohjelmointikielet
sallivat funktioiden rekursion ja tekevät usein siitä myös vaivatonta.\cite{hudak} Funktio, joka laskee luvun n kertoman
voidaan toteutta esimerkiksi näin funktionaalisesti Haskell-kielessä:
\begin{minted}{haskell}
factorial :: Int -> Int
factorial n = if n < 2
                then 1
                else n * factorial (n-1)
\end{minted}
% add reasons why this is a good idea, i.e. side-effects

\subsection{Korkeamman asteen funktiot}
Funktionaaliset ohjelmointikielet rohkaisevat ohjelmoimaan funktionaalisesti muun muassa sallimalla korkeamman asteen
funktiot. Tämä tarkoittaa, että funktioita kohdellaan ensimmäisen luokan kansalaisina ja niitä voi syöttää parametrina
funktioille, asettaa paluuarvoksi funktiolle ja tallentaa tietorakenteisiin. Korkeamman asteen funktioiden avulla
ohjelmakoodi ja data ovat jossain määrin vaihdettavissa, joten niiden avulla voidaan abstrahoida kompleksisia
rakenteita.\cite{hudak} Map-funktio on korkeamman asteen funktio, joka suorittaa parametrina annetun funktion jollekkin
tietorakennteelle, esimerkiksi listalle. Map-funktiota käytetään usein modernissa front-end web-ohjelmoinnissa
\cite{functionalreact}. Listan renderöinti React-kirjastolla voidaan toteuttaa näin map-funktion avulla:
\begin{minted}{jsx}
items.map(item => <Item {...item} />)
\end{minted}

\subsection{Laiska laskenta}
Laiska laskenta (eng. lazy evaluation) on funktion laskentastrategia, jossa lausekkeen arvo lasketaan vasta kun sitä
ensimmäisen kerran tarvitaan, mutta ei aikaisemmin. Tämä voi vähentää suoritukseen kuluvaa aikaa ja siten voi parantaa
ohjelman suorituskykyä eksponentiaalisesti esimerkiksi call by name -laskentastrategiaan verrattuna, jossa lausekkeen
arvo voidaan joutua laskemaan useita kertoja. Innokas laskenta (eng. eager evaluation) tarkoittaa lausekkeen arvon
laskemista heti ensimmäisellä kerralla, kun lauseke esitellään. Tämä on yleisimpien ohjelmointikielten
laskentastrategia. Laiskan laskennan toteuttavat useat puhtaasti funktionaaliset ohjelmointikielet, kuten Haskell. Myös
jotkin moniparadigmaiset kielet toteuttavat laiskan laskennan, esimerkiksi Scala-kielen \textit{lazy val}
-lauseke.\cite{languagedesign}

Laiska laskenta mahdollistaa päättymättömät tietorakenteet. Niin sanottujen laiskojen listojen loppupää evaluoidaan
vasta kun sitä kutsutaan. Tämä näkyy useissa funktionaalisissa ohjelmointikielissä listan toteutuksena, jossa listaa
evaluoidaan alkupäästä loppua kohti. Esimerkiksi Haskellissa voidaan määrittää lista kaikista kokonaisluvuista alkaen
luvusta \textit{n}:
\begin{minted}{haskell}
from :: Int -> [Int]
from n = n : from (n+1)
\end{minted}
Funktion rekursiivinen kutsu aiheuttaisi innokkaasti laskevissa ohjelmointikielissä päättymättömän rekursion, mutta
Haskellin tapauksessa vain määrätty osa listasta evaluoidaan.\cite{languagedesign}

Funktionaalisissa ohjelmointikielissä rekursio ei aiheuta pinon ylivuotoa, joten päättymätön rekursio on toimiva
vaihtoehto päättymättömälle silmukalle. Esimerkiksi Haskell-kielen main-funktioita voidaan toistaa äärettömästi näin:
\begin{minted}{haskell}
main :: IO ()
main =
  do
    putStrLn "do something"
    main
\end{minted}

\subsection{Hahmonsovitus}
Hahmonsovitus (eng. pattern matching) on puhtaalle funktionaaliselle ohjelmoinnille tyypillinen piirre, jossa sama
funktio voidaan määritellä useita kertoja. Funktiomäärittelyistä vain yhtä sovelletaan tapauskohtaisesti. Modernit
funktionaaliset ohjelmointikielet tekevät hahmonsovitustoteutuksistaan mahdollisimman ilmaisuvoimaisia rohkaistakseen
sen käyttöä. Hahmonsovituksen toteutus on käytännössä case-lauseke, jossa lausekkeen ehto kuvaa syötetyn parametrin
tietotyyppiä tai sen rakenneta. Tämän avulla ohjelman suoritus jakautuu syötetyn muuttujan tyypin tai rakenteen mukaan.
\cite{hudak}

Esimerkiksi kertomafunktio voidaan toteuttaa rekursiivisesti hahmonsovituksen avulla.
\begin{minted}{haskell}
factorial :: Int -> Int
factorial 0 = 1
factorial n = n * factorial (n - 1)
\end{minted}

\section{Funktionaalisen laskentamallin seuraukset}
Puhtaasti funktionaalinen laskentamalli karsii ohjelmasta pois tietyt bugityypit. Puhdas sivuvaikutukseton laskenta on
automaattisesti säieturvallista (eng. thread safe), mikä tekee ohjelman rinnakkaistamisesta helppoa. Puhtaan funktion
deterministisyys tarkoittaa, ettei muu samanaikainen suoritus voi vaikuttaa laskennnan oikeuteen, eikä sivuvaikutukseton
funktio myöskään pysty vaikuttamaan muun samanaikaisen laskennan lopputulokseen.

Deterministisyys ja toistettavuus parantavat ohjelman testattavuutta. Jos deterministinen funktio toimii tietyillä
parametreilla kerran, se toimii myös joka kerta ajon aikana. Toistettavuus helpottaa ajon aikana löydetyn virheellisen
tilan toistamista syöttämällä sama tarkkaan määritetty virheellinen tila testiympäristöön.\cite{functionaljava}

Tilattomuus ja abstraktit datatyypit käyttävät tyypillisesti enemmän prosessoritehoa ja muistia, kuin imperatiiviset
ohjelmat. Pahimmassa tapauksessa tämä aiheuttaa $ O(\log{}n) $ hakuajan funktionaaliselle datatyypille, verrattuna 
imperatiiviselle ohjelmoinnille tyypilliseen $ O(n) $. Kuitenkin ohjelmat, jotka tekevät intensiivisiä numeerisia 
laskuja, ovat vain marginaalisesti hitaampia funktionaalisilla ohjelmointikielillä kuten OCaml ja Clean verrattuna 
C-kieleen.\cite{benchmark}

\section{Ohjelmointikielet}
% - Lisp vs ML
% - Modern functional languages. Clojure, Haskell etc

% Kääntäminen lyhyesti
%   Ei lambdakalkyylin kautta vaan suoraan konekielelle

%\input{file_name_of_chapter_x}
%\input{file_name_of_chapter_y}

\begin{comment}
The thesis main content ends here.
\end{comment}
\printbibliography

\begin{comment}
main document ends here
\end{comment}

\end{document}
