\documentclass[a4paper,12pt,language=english,version=draft,hidechapters=true,includereferences=false,realtimesnewroman=false,sharelatex=false,emptyfirstpages=true]{utuftthesis}
\setcounter{secnumdepth}{2}
\setcounter{tocdepth}{2}

\addbibresource{references.bib}
\begin{document}
\begin{comment}
Document template suitable for use as a LaTeX master-file for master's
thesis in University of Turku Department of Future Technologies.\\
\\
Compatible with: ShareLaTeX / PDFLaTeX / XeLaTeX.\\
\\
\\
{*}{*} HOW TO USE? {*}{*}\\
\\
Want to write a thesis? Clone this template in ShareLaTeX or fork
the department's thesis git project.\\
\\
The utuftthesis.cls defines a new thesis class, which is based on
the report class. It supports these new named parameters:

- paper: a4paper

- version: draft / final (default: draft) shows/hides {[}draft{]}
in the header

- language: finnish / english (default: finnish) affects the general
document appearance and hyphenation

- hidechapters: true / false (default: true) hides/shows the chapter/luku
text at the beginning of each Chapter

- includereferences: true / false (default: false) include reference
pages when calculating the total number of pages

- realtimesnewroman: true / false (default: false) use Times New Roman
instead of LaTeX fonts with XeLaTeX. Requires the font to be installed
on the system / provided in the document directory. Other fonts can
be defined with \textbackslash setmainfont.

- sharelatex: true / false (default: false) don't attempt to use (c)
system fonts, instead read them from the project repository

- emptyfirstpages: true / false (default: true) clear the headers/footers
for the 1st pages of text chapters

Traditionally the best places to learn (La)TeX are probably the manual
pages for each package http://www.ctan.org/ and http://www.ctan.org/tex-archive/info/lshort/english/lshort.pdf.
This new version (2.0) should be compatible with xelatex and biblatex
which means that all source files can freely use normal UTF-8 text
without resorting to \textquotedbl\textquotedbl legacy hacks\textquotedbl .\\
\\
Note that PDF/A requirements don't allow PDF links, but if you want
to provide a user friendly version of the thesis with links, use \textbackslash hyperref\\
\\
\\
{*}{*} Maintenance {*}{*}\\
\\
Workflow: https://gitlab.utu.fi/ttweb/thesis -> master .lyx document
exported as .tex documents -> repository content dumped to the sharelatex
project template

Want to fix something in the template? Send a merge.\\
\\
Relies on utuftthesis.cls for the document class definitions.
\end{comment}


\pubyear{2019}

\pubmonth{6}

\publaben{Software Engineering}

\pubtype{tkk}
\title{Functional programming in front-end web development}
\author{Konsta Purtsi}

\maketitle
\keywords{funktionaalinen ohjelmointi, funktionaalinen reaktiivinen ohjelmointi, web-ohjelmointi, frontend, React, Elm}
\keywordstwo{functional programming, functional reactive programming, web-programming, frontend, React, Elm}

\begin{abstract}
Tietotekniikan käyttö arkielämässä on yleistynyt viime vuosina räjähdysmäisesti. Etenkin verkkosivujen ja -sovellusten
käyttö on suositumpaa kuin koskaan. Kasvanut käyttö on lisännyt merkittäväst web-tekniikoille asetettuja vaatimuksia
muun muassa luotettavuuden ja tuottavuuden kannalta. 

Funktionaalinen ohjelmointi on matemaattisten funktioiden evaluointiin perustuva ohjelmointiparadigma. Funktionaalinen
ohjelmointi ei ole paradigmana uusi keksintö, mutta sen soveltaminen web-kehitykseen on nostanut päätään vasta viime
vuosina. Tämän tutkielman tavoitteena on selvittää, miten funktionaalinen ohjelmointi soveltuu frontend-web-kehitykseen.

Tutkielmassa selvitetään ensin funktionaalisen ohjelmoinnin konsepteja ja etuja. Tämän jälkeen tarkastellaan
web-kehitystä, ja miten funktionaalinen ohjelmointi soveltuu frontend-web-sovelluksen toteutukseen. Tutkielmassa
tarkastellaan tarkemmin kahta funktionaalista reaktiivista frontend-ohjelmointitekniikkaa: React-kirjastoa ja
Elm-ohjelmointikieltä.

Funktionaalinen ohjelmointi ja etenkin funktionaalinen reaktiivinen ohjelmointi on hyvin luonteva tapa toteuttaa
frontend web-sovellus. Tutkituista tekniikoista molemmat tukevat funktionaalista ohjelmointityyliä, mutta Elm ei
myöskään salli muuta tapaa toteuttaa ohjelma.
\end{abstract}


% empty pagestyle for table of contents etc.
% otherwise you'll get simple page style with roman page numbers
\pagestyle{empty}

% mandatory
\tableofcontents

% if you want a list of figures
%\listoffigures

% if you want a list of tables
%\listoftables

% 'list of acronyms'
%   - you may not need this at all
%   - create a chapter called List Of Acronyms (or whatever), which
%     should contain all your acronym definitions, e.g.
%     \chapter{List Of Acronyms}
%   - the secnumdepth trickery is needed because acronyms are as a
%     standard chapter and we are faking '\listofacronyms'
%
%\setcounter{secnumdepth}{-1}
%\input{your acronym chapter's file name}
%\setcounter{secnumdepth}{2}% setup page numbering, page counter, etc.%
\begin{comment}
The thesis starts here.

To better organize things, create a new tex file for each chapter
and input it below.

Avoid using the å, ä, ö or <space> characters in referred names and
underscores \_ in file names (may break hyperref).

Good luck!
\end{comment}

\chapter{Introduction} \label{Introduction}
% Max 2-3 pages

\section{Motivation}
% Why is this a thing / motivation speech

\section{Research question}
% Overview on functional programming
% Overview on web development

\section{Work methodology}
% Work methodology

\section{Thesis structure}
% Structure of the paper

\chapter{Functional programming} \label{Functional programming}

% Functional programming paradigm
% - Based on lambda calculus
% - A bit of history

% - The paradigm
Functional programming is a computer programming paradigm in which the evaluation of a computer program is carried out
through the evaluation of deeply nested mathematical functions. Declarative programming is a programming paradigm that
describes the program only as expressions or declarations instead of statements. Functional programming is a declarative
programming paradigm that describes the program as function declarations. Functional code avoids changing-state and
mutable data, which can make understanding a program easier.

Functional programming is largely based on lambda calculus, a formal system developed in the 1930s by Alonzo Church.
Many functional programming languages, such as Lisp, can be seen as abstractions on top of lambda calculus. \cite{hudak}
% - modern lambda calculus

\section{Concepts}
% * declarative
% * immutability
% * higher order functions
% * lazy evaluation

\section{Languages}
% - Lisp vs ML
% - Modern functional languages. Clojure, Haskell etc



%\input{file_name_of_chapter_x}
%\input{file_name_of_chapter_y}

\begin{comment}
The thesis main content ends here.
\end{comment}
\printbibliography

\begin{comment}
main document ends here
\end{comment}

\end{document}
