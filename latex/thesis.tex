\documentclass[a4paper,12pt,language=finnish,version=draft,hidechapters=true,includereferences=false,realtimesnewroman=false,sharelatex=false,emptyfirstpages=true]{utuftthesis}
\setcounter{secnumdepth}{2}
\setcounter{tocdepth}{2}

\addbibresource{Bibliografia.bib}
\usepackage[outputdir=../dist]{minted}
\begin{document}
\begin{comment}
Document template suitable for use as a LaTeX master-file for master's
thesis in University of Turku Department of Future Technologies.\\
\\
Compatible with: ShareLaTeX / PDFLaTeX / XeLaTeX.\\
\\
\\
{*}{*} HOW TO USE? {*}{*}\\
\\
Want to write a thesis? Clone this template in ShareLaTeX or fork
the department's thesis git project.\\
\\
The utuftthesis.cls defines a new thesis class, which is based on
the report class. It supports these new named parameters:

- paper: a4paper

- version: draft / final (default: draft) shows/hides {[}draft{]}
in the header

- language: finnish / english (default: finnish) affects the general
document appearance and hyphenation

- hidechapters: true / false (default: true) hides/shows the chapter/luku
text at the beginning of each Chapter

- includereferences: true / false (default: false) include reference
pages when calculating the total number of pages

- realtimesnewroman: true / false (default: false) use Times New Roman
instead of LaTeX fonts with XeLaTeX. Requires the font to be installed
on the system / provided in the document directory. Other fonts can
be defined with \textbackslash setmainfont.

- sharelatex: true / false (default: false) don't attempt to use (c)
system fonts, instead read them from the project repository

- emptyfirstpages: true / false (default: true) clear the headers/footers
for the 1st pages of text chapters

Traditionally the best places to learn (La)TeX are probably the manual
pages for each package http://www.ctan.org/ and http://www.ctan.org/tex-archive/info/lshort/english/lshort.pdf.
This new version (2.0) should be compatible with xelatex and biblatex
which means that all source files can freely use normal UTF-8 text
without resorting to \textquotedbl\textquotedbl legacy hacks\textquotedbl .\\
\\
Note that PDF/A requirements don't allow PDF links, but if you want
to provide a user friendly version of the thesis with links, use \textbackslash hyperref\\
\\
\\
{*}{*} Maintenance {*}{*}\\
\\
Workflow: https://gitlab.utu.fi/ttweb/thesis -> master .lyx document
exported as .tex documents -> repository content dumped to the sharelatex
project template

Want to fix something in the template? Send a merge.\\
\\
Relies on utuftthesis.cls for the document class definitions.
\end{comment}


\pubyear{2019}

\pubmonth{6}

\publaben{Software Engineering}
\publab{Ohjelmistotekniikka}

\pubtype{tkk}
\title{Funktionaalinen ohjelmointi front-end web-kehityksessä}
\author{Konsta Purtsi}

\maketitle
\keywords{funktionaalinen ohjelmointi, funktionaalinen reaktiivinen ohjelmointi, web-ohjelmointi, frontend, React, Elm}
\keywordstwo{functional programming, functional reactive programming, web-programming, frontend, React, Elm}

\begin{abstract}
Tietotekniikan käyttö arkielämässä on yleistynyt viime vuosina räjähdysmäisesti. Etenkin verkkosivujen ja -sovellusten
käyttö on suositumpaa kuin koskaan. Kasvanut käyttö on lisännyt merkittäväst web-tekniikoille asetettuja vaatimuksia
muun muassa luotettavuuden ja tuottavuuden kannalta. 

Funktionaalinen ohjelmointi on matemaattisten funktioiden evaluointiin perustuva ohjelmointiparadigma. Funktionaalinen
ohjelmointi ei ole paradigmana uusi keksintö, mutta sen soveltaminen web-kehitykseen on nostanut päätään vasta viime
vuosina. Tämän tutkielman tavoitteena on selvittää, miten funktionaalinen ohjelmointi soveltuu frontend-web-kehitykseen.

Tutkielmassa selvitetään ensin funktionaalisen ohjelmoinnin konsepteja ja etuja. Tämän jälkeen tarkastellaan
web-kehitystä, ja miten funktionaalinen ohjelmointi soveltuu frontend-web-sovelluksen toteutukseen. Tutkielmassa
tarkastellaan tarkemmin kahta funktionaalista reaktiivista frontend-ohjelmointitekniikkaa: React-kirjastoa ja
Elm-ohjelmointikieltä.

Funktionaalinen ohjelmointi ja etenkin funktionaalinen reaktiivinen ohjelmointi on hyvin luonteva tapa toteuttaa
frontend web-sovellus. Tutkituista tekniikoista molemmat tukevat funktionaalista ohjelmointityyliä, mutta Elm ei
myöskään salli muuta tapaa toteuttaa ohjelma.
\end{abstract}


% empty pagestyle for table of contents etc.
% otherwise you'll get simple page style with roman page numbers
\pagestyle{empty}

% mandatory
\tableofcontents

% if you want a list of figures
%\listoffigures

% if you want a list of tables
%\listoftables

% 'list of acronyms'
%   - you may not need this at all
%   - create a chapter called List Of Acronyms (or whatever), which
%     should contain all your acronym definitions, e.g.
%     \chapter{List Of Acronyms}
%   - the secnumdepth trickery is needed because acronyms are as a
%     standard chapter and we are faking '\listofacronyms'
%
%\setcounter{secnumdepth}{-1}
%\input{your acronym chapter's file name}
%\setcounter{secnumdepth}{2}% setup page numbering, page counter, etc.%
\begin{comment}
The thesis starts here.

To better organize things, create a new tex file for each chapter
and input it below.

Avoid using the å, ä, ö or <space> characters in referred names and
underscores \_ in file names (may break hyperref).

Good luck!
\end{comment}

\chapter{Johdanto} \label{Johdanto}
Tietotekniikan käyttö arkielämässä on yleistynyt viime vuosina räjähdysmäisesti. Etenkin verkkosivujen ja -sovellusten
käyttö on suositumpaa kuin koskaan. Kasvanut käyttö on lisännyt merkittäväst web-tekniikoille asetettuja vaatimuksia
muun muassa luotettavuuden ja produktiivisuuden kannalta. 

Funktionaalinen ohjelmointi on matemaattisten funktioiden evaluointiin perustuva ohjelmointiparadigma. Funktionaalinen
ohjelmointi ei ole paradigmana uusi keksintö, mutta sen soveltaminen web-kehitykseen on nostanut päätään vasta viime
vuosina. Tämän tutkielman tavoitteena on selvittää, miten funktionaalinen ohjelmointi soveltuu frontend-web-kehitykseen.

Tutkielma käsittelee funktionaalista ohjelmointia yleisellä tasolla ja lisäksi tutkii kahden modernin
funktionaalisreaktiivisen frontend-tekniikan toteutusta funktionaalisuudelle. Tutkittaviksi teknologioiksi on valittu
React-kirjasto ja Elm-ohjelmointikieli, jotka molemmat kääntyvät selaimen tulkittavaksi JavaScript-kieleksi.

Funktionaalista ohjelmoinnin konsepteja ja syntyä käsittelee tutkielman luku 2. Luvussa selvitetään myös hyötyjä
funktionaalisen ohjelmoinnin käytöstä.

Funktionaalista ohjelmointia web-kehityksessä ja frontend-sovelluksen toteutuksessa etenkin funktionaalisen reaktiivisen
ohjelmoinnin avulla käsittelevät luvut 3 ja 4.

\chapter{Funktionaalinen ohjelmointi} \label{Funktionaalinen ohjelmointi}

\section{Määritelmä}
Funktionaalinen ohjelmointi on deklaratiivinen ohjelmointiparadigma, jossa ohjelman suoritus tapahtuu sisäkkäisten
matemaattisten funktioiden evaluoimisena. Deklaratiivinen ohjelmointi on ohjelmointiparadigma, jossa kuvaillaan ohjelman
lopputulos tai tila. Deklaratiivisessa ohjelmoinnissa algoritmi kuvataan varsinaisen kontrollivuon sijaan globaalia
tilaa muokkaavin funktiokutsuin. Se on vastakohta perinteisemmälle imperatiiviselle ohjelmoinnille. Funktionaalinen
ohjelmakoodi pyrkii käsittelemään tilaa tarkemmin ja välttämään muuttuvia arvoja.\cite{hudak}

Puhtaasti funktionaalinen ohjelmointikieli ei salli lainkaan muuttuvaa tilaa tai datan muokkaamista. Tämä vaatii sen,
että funktioiden paluuarvoo riippuu vain ja ainoastaan funktion saamista parametreista, eikä mikään tila voi vaikuttaa
funktion paluuarvoon. Simppelien loogisten toimintojen lisäksi tosielämän ohjelmointikielet vaativat kuitenkin joitain
epäpuhtauksia ollakseen oikeasti hyödyllisiä.\cite{purelyFunctional}

\section{Lambdakalkyyli}
Funktionaalinen ohjelmointi perustuu vahvasti lambdakalkyyliin - formaalin laskennan malliin jonka kehitti Alonzo Church
vuonna 1928. Lambdakalkyyliä pidetään ensimmäisenä funktionaalisena ohjelmointikielenä, vaikka se keksittiinkin ennen
ensimmäisiä varsinaisia tietokoneita, eikä sitä pidettykään aikanaan ohjelmointikielenä. Useita moderneja
ohjelmointikieliä, kuten Lispiä, pidetään abstraktioina lambdakalkyylin päälle. Lambdakalkyyli voidaan kirjoittaa myös
$ \lambda $-kalkyyli kreikkalaisella lambda-kirjaimella.\cite{hudak}

Yksinkertaisimmillaan lambdakalkyyli koostuu vain kolmesta termistä: muuttujista, abstraktioista ja sovelluksista.
Muuttujat ovat yksinkertaisesti merkkejä tai merkkijonoja jotka kuvaavat jotain parametriä tai arvoa. Churchin
alkuperäinen lambdakalkyyli ei tuntenut muuttujien asettamista, ainoastaan arvojen syöttämisen parametrina. Hänen
lambdakalkyylinsä primitiivitermit olivat abstraktio ja sovellus:

\begin{center}
  \begin{tabular}{ |c|c|c| } 
    \hline
    Syntaksi & Nimi & Kuvaus \\ [0.5ex] 
    \hline
    $ \lambda x [ M ] $ & Abstraktio & Määrittelee funktion parametrille x toteutuksella M \\ 
    \hline
    $ \{ F \} ( X ) $ & Sovellus & suorittaa funktion F arvolla X \\ 
    \hline
  \end{tabular}
\end{center}\cite{lambdacalculus}

% - moderni lambdakalkyyli

\section{Konseptit}
% * declarative?

\subsection{Tilattomuus}
Deklaratiivisissa ohjelmointikielissä ei ole implisiittistä tilaa, mikä eroaa imperatiivisista kielistä, joissa tilaa
voi muokata lauseilla (eng. commands) ohjelmakoodissa. Tilattomuus tekee laskennasta mahdollista ainoastaan lausekkeilla
(eng. expression). Funktionaaliset ohjelmointikielet perustuvat lambdakalkyyliin ja siten käyttävät ja
lambda-abstraktioita kaiken laskennan perustana. Esimerkiksi logiikkaaohjelmoinnissa vastaava rakenne on relaatio.
Tilattomuus sallii vain muuttumattomat vakiot, mikä estää muuttujien arvon muuttamisen ja perinteiset silmukat.
Puhtaasti funktionaalisessa ohjelmoinnissa rekursio on ainoa tapa toteuttaa toisto. Esimerkiksi funktio, joka laskee
luvun \textit{n} kertoman voitaisiin toteuttaa imperatiivisesti Python-kielessä esimerkiksi näin:
\begin{minted}{python}
def factorial(n):
    result = 1
    while n >= 1:
        result = result * n
        n = n - 1
    return result
\end{minted}
Funktionaalisessa ohjelmointikielessä toistorakenne on toteutettava rekursion avulla. Funktionaaliset ohjelmointikielet
sallivat funktioiden rekursion ja tekevät usein siitä myös vaivatonta.\cite{hudak} Funktio, joka laskee luvun n kertoman
voidaan toteutta esimerkiksi näin funktionaalisesti Haskell-kielessä:
\begin{minted}{haskell}
factorial :: Int -> Int
factorial n = if n < 2
                then 1
                else n * factorial (n-1)
\end{minted}
% add reasons why this is a good idea, i.e. side-effects

\subsection{Korkeamman asteen funktiot}
Funktionaaliset ohjelmointikielet rohkaisevat ohjelmoimaan funktionaalisesti muun muassa sallimalla korkeamman asteen
funktiot. Tämä tarkoittaa, että funktioita kohdellaan ensimmäisen luokan kansalaisina ja niitä voi syöttää parametrina
funktioille, asettaa paluuarvoksi funktiolle ja tallentaa tietorakenteisiin. Korkeamman asteen funktioiden avulla
ohjelmakoodi ja data ovat jossain määrin vaihdettavissa, joten niiden avulla voidaan abstrahoida kompleksisia
rakenteita.\cite{hudak} Map-funktio on korkeamman asteen funktio, joka suorittaa parametrina annetun funktion jollekkin
tietorakennteelle, esimerkiksi listalle. Map-funktiota käytetään usein modernissa front-end web-ohjelmoinnissa
\cite{functionalreact}. Listan renderöinti React-kirjastolla voidaan toteuttaa näin map-funktion avulla:
\begin{minted}{jsx}
items.map(item => <Item {...item} />)
\end{minted}

\subsection{Laiska laskenta}
Laiska laskenta (eng. lazy evaluation) on funktion laskentastrategia, jossa lausekkeen arvo lasketaan vasta kun sitä
ensimmäisen kerran tarvitaan, mutta ei aikaisemmin. Tämä voi vähentää suoritukseen kuluvaa aikaa ja siten voi parantaa
ohjelman suorituskykyä eksponentiaalisesti esimerkiksi call by name -laskentastrategiaan verrattuna, jossa lausekkeen
arvo voidaan joutua laskemaan useita kertoja. Innokas laskenta (eng. eager evaluation) tarkoittaa lausekkeen arvon
laskemista heti ensimmäisellä kerralla, kun lauseke esitellään. Tämä on yleisimpien ohjelmointikielten
laskentastrategia. Laiskan laskennan toteuttavat useat puhtaasti funktionaaliset ohjelmointikielet, kuten Haskell. Myös
jotkin moniparadigmaiset kielet toteuttavat laiskan laskennan, esimerkiksi Scala-kielen \textit{lazy val}
-lauseke.\cite{languagedesign}

Laiska laskenta mahdollistaa päättymättömät tietorakenteet. Niin sanottujen laiskojen listojen loppupää evaluoidaan
vasta kun sitä kutsutaan. Tämä näkyy useissa funktionaalisissa ohjelmointikielissä listan toteutuksena, jossa listaa
evaluoidaan alkupäästä loppua kohti. Esimerkiksi Haskellissa voidaan määrittää lista kaikista kokonaisluvuista alkaen
luvusta \textit{n}:
\begin{minted}{haskell}
from :: Int -> [Int]
from n = n : from (n+1)
\end{minted}
Funktion rekursiivinen kutsu aiheuttaisi innokkaasti laskevissa ohjelmointikielissä päättymättömän rekursion, mutta
Haskellin tapauksessa vain määrätty osa listasta evaluoidaan.\cite{languagedesign}

Funktionaalisissa ohjelmointikielissä rekursio ei aiheuta pinon ylivuotoa, joten päättymätön rekursio on toimiva
vaihtoehto päättymättömälle silmukalle. Esimerkiksi Haskell-kielen main-funktioita voidaan toistaa äärettömästi näin:
\begin{minted}{haskell}
main :: IO ()
main =
  do
    putStrLn "do something"
    main
\end{minted}

\subsection{Hahmonsovitus}
Hahmonsovitus (eng. pattern matching) on puhtaalle funktionaaliselle ohjelmoinnille tyypillinen piirre, jossa sama
funktio voidaan määritellä useita kertoja. Funktiomäärittelyistä vain yhtä sovelletaan tapauskohtaisesti. Modernit
funktionaaliset ohjelmointikielet tekevät hahmonsovitustoteutuksistaan mahdollisimman ilmaisuvoimaisia rohkaistakseen
sen käyttöä. Hahmonsovituksen toteutus on käytännössä case-lauseke, jossa lausekkeen ehto kuvaa syötetyn parametrin
tietotyyppiä tai sen rakenneta. Tämän avulla ohjelman suoritus jakautuu syötetyn muuttujan tyypin tai rakenteen mukaan.
\cite{hudak}

Esimerkiksi kertomafunktio voidaan toteuttaa rekursiivisesti hahmonsovituksen avulla.
\begin{minted}{haskell}
factorial :: Int -> Int
factorial 0 = 1
factorial n = n * factorial (n - 1)
\end{minted}

\section{Funktionaalisen laskentamallin seuraukset}
Puhtaasti funktionaalinen laskentamalli karsii ohjelmasta pois tietyt bugityypit. Puhdas sivuvaikutukseton laskenta on
automaattisesti säieturvallista (eng. thread safe), mikä tekee ohjelman rinnakkaistamisesta helppoa. Puhtaan funktion
deterministisyys tarkoittaa, ettei muu samanaikainen suoritus voi vaikuttaa laskennnan oikeuteen, eikä sivuvaikutukseton
funktio myöskään pysty vaikuttamaan muun samanaikaisen laskennan lopputulokseen.

Deterministisyys ja toistettavuus parantavat ohjelman testattavuutta. Jos deterministinen funktio toimii tietyillä
parametreilla kerran, se toimii myös joka kerta ajon aikana. Toistettavuus helpottaa ajon aikana löydetyn virheellisen
tilan toistamista syöttämällä sama tarkkaan määritetty virheellinen tila testiympäristöön.\cite{functionaljava}

Tilattomuus ja abstraktit datatyypit käyttävät tyypillisesti enemmän prosessoritehoa ja muistia, kuin imperatiiviset
ohjelmat. Pahimmassa tapauksessa tämä aiheuttaa $ O(\log{}n) $ hakuajan funktionaaliselle datatyypille, verrattuna 
imperatiiviselle ohjelmoinnille tyypilliseen $ O(n) $. Kuitenkin ohjelmat, jotka tekevät intensiivisiä numeerisia 
laskuja, ovat vain marginaalisesti hitaampia funktionaalisilla ohjelmointikielillä kuten OCaml ja Clean verrattuna 
C-kieleen.\cite{benchmark}

\section{Ohjelmointikielet}
% - Lisp vs ML
% - Modern functional languages. Clojure, Haskell etc

% Kääntäminen lyhyesti
%   Ei lambdakalkyylin kautta vaan suoraan konekielelle
\chapter{Web-kehitys}

\section{World Wide Web}
World Wide Web tai WWW on järjestelmä tiedon jakamiseen, joka käyttää Internet-verkkoa. Web-sisältöä luetaan selaimella,
joka hakee sisällön web-palvelimelta HTTP-siirtoprotokollan avulla. Verkkosivujen sisällön kuvaamiseen on perinteiseti
ollut kolme tekniikkaa: HTML-kieli sivun sisällön kuvaamiseen, CSS-kieli sivun ulkoasun kuvaamiseen ja JavaScript-kieli
sivun toiminnallisuuden toteuttamiseen. On myös mahdollista käyttää esimerkiksi Flash- ja Java-liitännäisiä, mutta
nykyään on tyypillisempää käyttää pelkästään JavaScript-kieltä. \cite{javascriptguide}

\subsection{Staattiset verkkosivut}
Alkuperäinen tapa jakaa web-sisältöä on staattisena verkkosivuna (eng. static website). Staattinen verkkosivu koostuu
HTML-sivuista, dokumenteista ja mediasta, jotka luetaan suoraan web-palvelimen massamuistista, ilman että palvelin tekee
niihin muutoksia ajon aikana. Staattiset verkkosivut sisältävät usein HTML-, CSS- ja multimediasisältöä, sekä selaimessa
ajettavia skriptejä tai ohjelmia, jotka on kirjoitettu JavaScript-kielellä tai muulla selaimen tulkitsemalla kielellä.
\cite{staticdynamicwebsites}

\subsection{Dynaamiset verkkosivut}
Dynaaminen verkkosivu tarkoittaa sivua, joka ei ole suoraan selaimen luettavassa muodossa. Dynaamisen sivun back-end
muodostaa selaimella renderöityvän sivun ajon aikana. Dynaamisen verkkosivun operaatioiden toteutus ei olekaan rajattu
selaimen tulkittaviin ohjelmointikieliin, vaan käytännössä minkä tahansa kielen käyttäminen on mahdollista. Yleisimpiä
kieliä dynaamisen verkkosivun toteuttamiseen ovat PHP, Python, Perl, Ruby, Java, C\# ja Node.js. Dynaaminen verkkosivu
tallettaa useimmiten sisältönsä tietokantaan. \cite{staticdynamicwebsites}

\section{Back-end}
Back-end tarkoittaa verkkosivun tai -sovelluksen palvelinpäässä ajettavaa osaa, joka ei näy käyttäjälle. Koska
back-end-ohjelmaa ajetaan palvelinlaitteistolla, sen toteuttamiseen on paljon enemmän vapautta kuin front-endin, jonka
täytyy olla selaimen tulkittavissa. Back-end vastaa usein kaikesta varsinaisesta toiminnallisuudesta verrattuna
front-endiin, joka toteutaa pelkän käyttöliittymän ohjelmalle. Back-endiin kuuluu usein myös tietokanta, johon
sovelluksen data tallennetaan. \cite{fullstackdeveloper}

Back-end kommunikoi front-endin kanssa jonkin yhteisen rajapinnan (eng. Application programming interface, API) kautta.
Rajapinta on yleensä pääasiallinen tiedonsiirtokanava back-endin ja front-endin välillä. Front-end lähettää rajapintaan
kutsun useimmiten HTTP-protokollan avulla, palvelin vastaanottaa kutsun ja suorittaa kutsutun toimenpiteen.
Rajapintaa suunniteltaessa tulee muotoilla rajapinnan rakenne vastaamaan sekä ohjelman datarakennetta, että käyttäjän
tai front-endin vaatimuksia. Rajapinta tarjoaa staattiselle front-endille mahdollisuuden dynaamiseen ajonaikana
haettavaan sisältöön ja sen muokkaamiseen. Rajapinta voi tarjota front-endille käyttäjän tunnistautumisen (eng.
authentication) ja valtuuttamisen (eng. authorization). \cite{fullstackdeveloper}

\section{Front-end}
Front-end tarkoittaa verkkosivun tai -sovelluksen käyttäjälle näkyvää osaa. Koska front-endiä suoritetaan selaimessa,
se on pakko toteuttaa web-yhteensopivilla tekniikoilla, joista yleisimmät ovat HTML, CSS ja JavaScript. JavaScript on
nykypäivänä täysiverinen ohjelmointikieli, jolla voidaan toteuttaa interaktiivisuutta sekä manipuloida verkkosovelluksen
rakennetta ja ulkoasua. \cite{fullstackdeveloper}

Front-endin keskeisin tehtävä on käyttöliittymän ja tiedon näyttäminen käyttäjälle. Tästä huolimatta modernissa
front–endissä on paljon muitakin asioita joita tulee ottaa huomioon. Front-end ohjelmakoodi muodostaa ja muokkaa
verkkosivun HTML DOM-puuuta (eng. DOM tree), joka kuvaa sivun rakenteen. Front-end mahdollistaa myös käyttäjält
tulevien DOM-tapahtumien (eng. DOM event) vastaanottamisen ja käsittelyn. Modernilta front-endiltä odotetaan myös
responsiivista designia, joka tarkoittaa että sama sivu toimii usealla erikokoisella päätelaitteella.
\cite{bignerdfrontend}

Modernissa front-end–kehityksessä on tyypillistä toteuttaa verkkosovelluksen ohjelmakoodi jollain korkeamman tason
ohjelmointikiellä, joka muunnetaan web-natiiviksi käännösvaiheessa. Usein käytetään esimerkiksi CSS-tyylikieleksi
kääntyvää SCSS-kieltä. Kääntäjiä, jotka kääntävät front-end–kieliä selainyhteensopiviksi kutsutaan käännöstyökaluiksi
(eng. build tool). Esimerkiksi tämän tutkielman käsittelykohteena oleva Elm-kieli ei ole sellaisenaan
selainyhteensopiva, vaan vaatii kääntämisen HTML- ja JavaScript-kielille \cite{elmlang}. \cite{fullstackdeveloper}

\section{Funktionaalinen web-ohjelmointi}
Funktionaalinen web-ohjelmointi on kerännyt suosiota viime vuosina. Funktionaalinen ohjelmointi soveltuu erinomaisesti
web-ohjelmointiin sen toistettavuuden ja testattavuuden ansiosta. Bisneslogiikan, tietokantalogiikan ja
ohjelmistokehyskoodin (eng. framework boilerplate) yksinkertaisempi erottelu on myös perustelu funktionaaliselle
web-ohjelmoinnille. Back-end sovelluksen ohjelmointikieleksi voikin valita esimerkiksi Clojure- tai Elixir-kielen.
\cite{functionalwebdev} Funktionaalinen ohjelmointi on yleistä etenkin front-end kehityksessä funktionaalisia piirteitä
suosivien tekniikoiden, kuten React, ansiosta. Funktionaalinen front-end–ohjelmointi tuo tyypillisiä funktionaalisen
ohjelmoinnin hyötyjä, kuten parantaa ohjelman suorituskykyä ja testattavuutta. \cite{functionalreact} Koska tutkielman
aihe on funktionaalinen front-end web-kehitys, käsitellään seuraavaksi enemmän front-endia.

\begin{itemize}
  \item määrittele työssä käsiteltävät asiat
  \item yksityiskohtaisempaa kuvausta kappaleeseen 4
  \item asioita jotka ovat molemmille yhteisiä
  \item miten tilaa käsitellään front-endissä funktionaalisesti esim. reducerin ja modelin konseptit (tutkimuskohde 1)
  \item miten frameworkit mahdollistavat sivuvaikutukset (tutkimuskohde 2)
  \item Elmissä on sivuvaikutuksia, koska innokas laskenta
\end{itemize}

\subsubsection{Funktionaalinen reaktiivinen ohjelmointi}

// En oo ihan varma minkä tason otsikon tä tarvii. Tosi keskeistä kamaa kuitenkin

\begin{itemize}
  \item mitä tarkoittaa
  \item mitä konsepteja molemmat frameworkit sisältävät
  \item miten framework käsittelee reaktiivisesti esim. eventit, api-kutsut (tutkimuskohde 3)
\end{itemize}

\chapter{Funktionaaliset front-end ohjelmistokehykset}
Tämän tutkielman tutkimuskohteiksi on valittu funktionaaliset reaktiiviset ohjelmistokehykset React-kirjasto ja
Elm-ohjelmointikieli. React-kirjastoa ja Elm-kieltä tutkitaan yleisesti funktionaalisen ohjelmoinnin kannalta, sekä
tarkemmin siitä, miten ne toteuttavat tilankäsittelyn, mahdollistavatko ne sivuvaikutukset ja miten toteuttavat
asynkroniset tapahtumat.

\section{Ohjelmistokehysten esittely}
React on Facebookin ylläpitämä avoimen lähdekoodin käyttöliittymäkirjasto, jota käytetään tyypillisesti
JavaScript-kielen kanssa. React on alunperin julkaistu vuonna 2013 ja se on kirjoitushetkellä vuonna 2020 maailman
suosituin front-end ohjelmistokehys \cite{npmtrends}. React-kirjaston syntaksi on deklaratiivista, ja se mahdollistaa
käyttöliittymän pilkkomisen komponentteihin. React ei ota kantaa muihin tekniikkaratkaisuihin, vaan keskittyy pelkästään
yksittäisen DOM-puun hallitsemiseen. Ohjelmistokehittäjälle jää näin täysi valta valita muut teknologiat vapaasti.
\cite{reactjs}

Elm on funktionaalinen ohjelmointikieli, joka käännetään JavaScript-kieleksi. Elmin on alunperin kehittänyt Evan
Czaplicki osana maisterin tutkielmaansa (eng. Master's thesis) vuonna 2012. Elm ei ole tutkielman kirjoitushetkellä
saavuttanut suurta suosiota, vaan sen NPM-paketinhallinnan latausmäärä on vain murto-osa React-kirjaston latausmäärästä 
\cite{npmtrends}. Elm keskittyy web-pohjaisten graafisten käyttöliittymien deklaratiiviseen toteuttamiseen. Elm
kytkeytyy tavallisen HTML DOM-puun solmuun. Elm ei myöskään ota kantaa muihin tekniikkaratkaisuihin, mutta sen kanssa on
mahdotonta käyttää muilla kielillä kirjoitettuja komponentteja ja kirjastoja. \cite{elmlang}

\section{Funktionaalinen ohjelmointi}

// miten mahdollistavat funktionaalista ohjelmointia

\section{Funktionaalinen tilankäsittely}

// Reactissa: useState, useReducer, Redux

// Elmissä: Model-abstraktio

\section{Sivuvaikutukset}

// Reactissa: imperatiivinen ohjelmointi, useEffect

// Elmissä: Elm-runtime abstraktoi, innokas laskenta aiheuttaa sivuvaikutuksia

\section{Funktionaalinen tapahtumankäsittely}

// eventit, api-kutsut \& asynkronisuus

\chapter{Loppupäätelmät}
Funktionaalinen ohjelmointi on kasvattanut suosiotaan etenkin front-end-ohjelmoinnissa viime vuosina.
Funktionaalisesta ohjelmoinnista onkin paljon hyötyjä verrattuna perinteiseen imperatiiviseen ohjelmointiin. Puhtaan
funktionaalisen ohjelman tilattomuus tekee ohjelmasta helpommin järkeiltävää ja testattavampaa. Puhdas funktio on myös
automaattisesti säieturvallinen. Funktionaaliselle ohjelmointityylille on myös tyypillistä koodin kompaktiutta ja
luettavuutta parantavien abstraktioiden, kuten hahmonsovitus, käyttö.

Funktionaalinen paradigma soveltuu web-kehitykseen erinomaisesti, kuten monen funktionaalisen kielen suosiosta voi
päätellä. Back-end-ohjelmoinnissa suosittuja kieliä ovat muun muassa Clojure- ja Elixir-kielet. Front-end-ohjelmoinnissa
funktionaalinen ohjelmointityyli on merkittävästi suositumpaa, muun muassa React-kirjaston myötä. Funktionaalinen
responsiivinen ohjelmointi on hyvin tyypillinen tapa toteuttaa moderni front-end-verkkosovellus.

Tässä tutkielmassa esitellyt teknologiat, React-kirjasto ja Elm-ohjelmointikieli toteuttavat molemmat funktionaalista
ohjelmointiparadigmaa. Näistä etenkin React on saavuttanut valtavan suosion, ja onkin maailman suosituin front-end
ohjelmistokehys. Sekä React, että Elm mahdollistavat funktionaalisen ohjelmoinnin piirteiden, kuten sivuvaikutusten
ja funktionaalisen tilanhallinnan toteuttamisen.

%\input{file_name_of_chapter_x}
%\input{file_name_of_chapter_y}

\begin{comment}
The thesis main content ends here.
\end{comment}
\printbibliography

\begin{comment}
main document ends here
\end{comment}

\end{document}
