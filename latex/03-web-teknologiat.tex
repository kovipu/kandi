\chapter{Web-kehitys}

\section{World Wide Web}
World Wide Web tai WWW on järjestelmä tiedon jakamiseen, joka käyttää Internet-verkkoa. Web-sisältöä luetaan selaimella,
joka hakee sisällön web-palvelimelta HTTP-siirtoprotokollan avulla. Verkkosivujen sisällön kuvaamiseen on perinteiseti
ollut kolme tekniikkaa: HTML-kieli sivun sisällön kuvaamiseen, CSS-kieli sivun ulkoasun kuvaamiseen ja JavaScript-kieli
sivun toiminnallisuuden toteuttamiseen. On myös mahdollista käyttää esimerkiksi Flash- ja Java-liitännäisiä, mutta
nykyään on tyypillisempää käyttää pelkästään JavaScript-kieltä. \cite{javascriptguide}

\subsection{Staattiset verkkosivut}
Alkuperäinen tapa jakaa web-sisältöä on staattisena verkkosivuna (eng. static website). Staattinen verkkosivu koostuu
HTML-sivuista, dokumenteista ja mediasta, jotka luetaan suoraan web-palvelimen muistista, ilman että palvelin tekee
niihin muutoksia ajon aikana. Staattiset verkkosivut sisältävät usein HTML-, CSS- ja multimediasisältöä, sekä selaimessa
ajettavia skriptejä tai ohjelmia, jotka on kirjoitettu JavaScript-kielellä tai muulla selaimen tulkitsemalla kielellä.
\cite{staticdynamicwebsites}

\subsection{Dynaamiset verkkosivut}
Dynaaminen verkkosivu tarkoittaa sivua, joka ei ole suoraan selaimen luettavassa muodossa. Dynaamisen sivun back-end
muodostaa selaimella renderöityvän sivun ajon aikana. Dynaamisen verkkosivun operaatioiden toteutus ei olekaan rajattu
selaimen tulkittaviin ohjelmointikieliin, vaan käytännössä minkä tahansa kielen käyttäminen on mahdollista. Yleisimpiä
kieliä dynaamisen verkkosivun toteuttamiseen ovat PHP, Python, Perl, Ruby, Java, C\# ja Node.js. Dynaaminen verkkosivu
tallettaa useimmiten sisältönsä tietokantaan. \cite{staticdynamicwebsites}

\section{Back-end}
Back-end tarkoittaa verkkosivun tai -sovelluksen palvelinpäässä ajettavaa osaa, joka ei näy käyttäjälle. Back-end
kommunikoi front-endin kanssa jonkin yhteisen rajapinnan (eng. Application programming interface, API) kautta. Koska
backend-ohjelmaa ajetaan palvelinlaitteistolla, sen toteuttamiseen on paljon enemmän vapautta kuin front-endin, jonka
täytyy olla selaimen tulkittavissa. Back-end vastaa usein kaikesta varsinaisesta toiminnallisuudesta verrattuna
front-endiin, joka toteutaa pelkän käyttöliittymän ohjelmalle. Backendiin kuuluu usein myös tietokanta, johon
sovelluksen data tallennetaan. \cite{fullstackdeveloper}

\begin{itemize}
  \item miten liittyy front-endin kanssa elämiseen
  \item API:t lyhyesti
  \item dynaaminen sisältö
\end{itemize}

\section{Front-end}
Front-end tarkoittaa verkkosivun tai -sovelluksen käyttäjälle näkyvää osaa. Koska front-endiä suoritetaan selaimessa,
se on pakko toteuttaa web-yhteensopivilla tekniikoilla, joista yleisimmät ovat HTML, CSS ja JavaScript. JavaScript on
nykypäivänä täysiverinen ohjelmointikieli, jolla voidaan toteuttaa interaktiivisuutta sekä manipuloida verkkosovelluksen
rakennetta ja ulkoasua. \cite{fullstackdeveloper}

// Tähän tekstiä ainakin: eli siis mitä frontissa oikeesti tapahtuu? eventit, responsiivisuus yms

\subsection{Käännöstyökalut}
Modernissa front-end–kehityksessä on tyypillistä toteuttaa verkkosovelluksen ohjelmakoodi jollain korkeamman tason
ohjelmointikiellä, joka muunnetaan web-natiiviksi käännösvaiheessa. Usein käytetään esimerkiksi CSS-tyylikieleksi
kääntyvää SCSS-kieltä. Kääntäjiä, jotka kääntävät front-end–kieliä selainyhteensopiviksi kutsutaan käännöstyökaluiksi
(eng. build tool). Esimerkiksi tämän tutkielman käsittelykohteena oleva Elm-kieli ei ole sellaisenaan
selainyhteensopiva, vaan vaatii kääntämisen HTML- ja JavaScript-kielille \cite{elmlang}. \cite{fullstackdeveloper}

\section{Funktionaalinen ohjelmointi web-ohjelmoinnissa}

\begin{itemize}
  \item yleisiä periaatteita
  \item mitä tällä tehdään
  \item koska tutkielman aihe on front-end, kerron seuraavaksi front-endistä enemmän
\end{itemize}

\subsection{Front-end kehityksessä}

\begin{itemize}
  \item määrittele työssä käsiteltävät asiat
  \item yksityiskohtaisempaa kuvausta kappaleeseen 4
  \item asioita jotka ovat molemmille yhteisiä
  \item miten tilaa käsitellään front-endissä funktionaalisesti esim. reducerin ja modelin konseptit (tutkimuskohde 1)
  \item miten frameworkit mahdollistavat sivuvaikutukset (tutkimuskohde 2)
  \item Elmissä on sivuvaikutuksia, koska innokas laskenta
\end{itemize}

\subsubsection{Funktionaalinen reaktiivinen ohjelmointi}

// En oo ihan varma minkä tason otsikon tä tarvii. Tosi keskeistä kamaa kuitenkin

\begin{itemize}
  \item mitä tarkoittaa
  \item mitä konsepteja molemmat frameworkit sisältävät
  \item miten framework käsittelee reaktiivisesti esim. eventit, api-kutsut (tutkimuskohde 3)
\end{itemize}
