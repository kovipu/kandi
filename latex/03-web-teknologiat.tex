\chapter{Web-kehitys}

\section{World Wide Web}
World Wide Web tai WWW on järjestelmä tiedon jakamiseen verkkosivuina ja -sovelluksina. Se hyödyntää maailmanlaajuista
Internet-tietoverkkoa. Web-sisältöä luetaan selaimella, joka hakee sisällön web-palvelimelta HTTP-siirtoprotokollan
avulla. Verkkosivujen sisällön kuvaamiseen on perinteiseti ollut kolme tekniikkaa: HTML-kieli sivun sisällön
kuvaamiseen, CSS-kieli sivun ulkoasun kuvaamiseen ja JavaScript-kieli sivun toiminnallisuuden toteuttamiseen. On myös
mahdollista käyttää esimerkiksi Flash- ja Java-liitännäisiä, mutta nykyään on tyypillisempää käyttää pelkästään
JavaScript-kieltä. \cite{javascriptguide}

\subsection{Staattiset verkkosivut}
Alkuperäinen tapa jakaa web-sisältöä on staattisena verkkosivuna (engl. static website). Staattinen verkkosivu koostuu
HTML-sivuista, dokumenteista ja mediasta, jotka luetaan suoraan web-palvelimen massamuistista. Palvelin lähettää
tiedostot selaimelle tekemättä niihin muutoksia ajon aikana. Staattiset verkkosivut sisältävät usein HTML-, CSS- ja
multimediasisältöä, sekä selaimessa ajettavia komentosarjoja (engl. script) tai ohjelmia, jotka on kirjoitettu
JavaScript-kielellä tai muulla selaimen tulkitsemalla kielellä. \cite{staticdynamicwebsites}

\subsection{Dynaamiset verkkosivut}
Dynaaminen verkkosivu tarkoittaa sivua, joka ei ole suoraan selaimen luettavassa muodossa. Tällaisen sivun HTML-koodi
generoidaan lennosta verkkosivun palvelimella. Dynaamisen verkkosivun operaatioiden toteutus ei olekaan rajattu selaimen
tulkittaviin ohjelmointikieliin, vaan käytännössä minkä tahansa kielen käyttäminen on mahdollista. Yleisimpiä kieliä
dynaamisen verkkosivun toteuttamiseen ovat muun muassa Python, Java, C\# ja Node.js. Dynaaminen verkkosivu tallettaa
useimmiten sisältönsä tietokantaan. \cite{staticdynamicwebsites}

\section{Backend}
Backend tarkoittaa verkkosivun tai -sovelluksen palvelinpäässä ajettavaa osaa, joka ei näy käyttäjälle. Koska
backend-ohjelmaa ajetaan palvelinlaitteistolla, sen toteuttamiseen on paljon enemmän vapautta kuin frontendin, jonka
täytyy olla selaimen tulkittavissa. Backend vastaa usein kaikesta varsinaisesta toiminnallisuudesta verrattuna
frontendiin, joka toteuttaa pelkän käyttöliittymän ohjelmalle. Backendiin kuuluu usein myös tietokanta, johon sovelluksen
data tallennetaan. \cite{fullstackdeveloper}

Backend kommunikoi frontendin kanssa jonkin yhteisen rajapinnan (engl. Application programming interface, API) kautta.
Rajapinta on yleensä pääasiallinen tiedonsiirtokanava backendin ja frontendin välillä. Frontend lähettää rajapintaan
kutsun useimmiten HTTP-protokollan avulla, jonka jälkeen palvelin vastaanottaa kutsun ja suorittaa kutsutun
toimenpiteen. Rajapintaa suunniteltaessa tulee muotoilla rajapinnan rakenne vastaamaan sekä ohjelman datarakennetta,
että käyttäjän tai frontendin vaatimuksia. Rajapinta mahdollistaa staattiselle frontendille dynaamisen sisällön
hakemisen ja muokkaamisen ajonaikana. Rajapinta voi myös mahdollistaa frontendille käyttäjän tunnistautumisen (engl.
authentication) ja valtuuttamisen (engl. authorization). \cite{fullstackdeveloper}

\section{Frontend}
Frontend tarkoittaa verkkosivun tai -sovelluksen käyttäjälle näkyvää osaa. Koska frontendiä suoritetaan selaimessa,
se on pakko toteuttaa web-yhteensopivilla tekniikoilla. Yleisimpiä tekniikoita ovat HTML, CSS ja JavaScript. JavaScript
kehitettiin alunperin hyvin yksinkertaisiin käyttöliittymätoimintoihin, kuten animaatoiden toteuttamiseen. Kuitenkin
nykypäivänä se on täysi ohjelmointikieli, jolla voidaan toteuttaa interaktiivisuutta sekä manipuloida verkkosovelluksen
rakennetta ja ulkoasua. Dokumenttioliomalli tai DOM (engl. Document Object Model) kuvaa verkkosivun puurakenteena, jonka
pohjalta sivu piirretään (engl. rendering), eli muodostetaan varsinainen kuva datamallien pohjalta.
\cite{fullstackdeveloper}

Frontendin keskeisin tehtävä on käyttöliittymän ja tiedon näyttäminen käyttäjälle. Tästä huolimatta modernissa
front–endissä on paljon muitakin asioita joita tulee ottaa huomioon. Frontend-ohjelmakoodi muodostaa ja muokkaa
verkkosivun HTML DOM-puuuta (engl. DOM tree), joka kuvaa sivun rakenteen. Frontend mahdollistaa myös käyttäjältä
tulevien DOM-tapahtumien (engl. DOM event) vastaanottamisen ja käsittelyn. Modernilta frontendiltä odotetaan myös
responsiivista designia, joka tarkoittaa että sama sivu toimii usealla erikokoisella päätelaitteella.
\cite{bignerdfrontend}

Modernissa frontend–kehityksessä on tyypillistä toteuttaa verkkosovelluksen ohjelmakoodi jollain korkeamman tason
ohjelmointikiellä, joka muunnetaan selaimen ymmärtämään muotoon käännösvaiheessa. Usein käytetään esimerkiksi
CSS-tyyli\-kieleksi kääntyvää SCSS-kieltä. Kääntäjiä, jotka kääntävät frontend–kieliä selainyhteensopiviksi kutsutaan
käännöstyökaluiksi (engl. build tool). Esimerkiksi tämän tutkielman käsittelykohteena oleva Elm-kieli ei ole
sellaisenaan selainyhteensopiva, vaan vaatii kääntämisen HTML- ja JavaScript-kielille \cite{elmlang}.
\cite{fullstackdeveloper}

\section{Funktionaalinen web-ohjelmointi}
Funktionaalinen web-ohjelmointi on kerännyt suosiota viime vuosina. Funktionaalinen ohjelmointi soveltuu erinomaisesti
web-ohjelmointiin sen toistettavuuden ja testattavuuden ansiosta. Eri abstraktiotasojen selkeämpi erottelu on myös
perustelu funktionaaliselle web-ohjelmoinnille. Backend-sovelluksen ohjelmointikieleksi voikin valita esimerkiksi
Clojure- tai Elixir-kielen. \cite{functionalwebdev} Funktionaalinen ohjelmointi on yleistä etenkin frontend-kehityksessä
funktionaalisia piirteitä suosivien tekniikoiden, kuten Reactin, ansiosta. Funktionaalinen frontend–ohjelmointi
mahdollistaa tyypillisiä funktionaalisen ohjelmoinnin hyötyjä, kuten parantaa ohjelman testattavuutta.
\cite{functionalreact} Koska tutkielman aihe on funktionaalinen frontend-web-kehitys, käsitellään seuraavaksi
frontendia.

Funktionaalista paradigmaa toteuttava frontend perustuu datan ja näkymän yhdistämiseen puhtaiden funktioiden avulla.
Tämänkaltainen frontend vaatii jonkin eksplisiittisen mekanismin datan muuttamiseen ja näkymän piirtämiseen uudelleen.
Funktionaaliset frontend-tekniikat suosivatkin usein muuttumatonta (engl. immutable) tilanhallintaa, joka vaatii tilan
uudelleenasettamisen sen muuttamiseksi. Tämänkaltaiselle tilanhallinnalle on myös tyypillistä säilöä koko ohjelman tila
yhdessä paikassa, ja muuttaa sitä lähettämällä sille käskyjä (engl. dispatching actions). Käskyn vastaanottaa puhdas
funktio, joka ottaa parametreina käskyn ja aiemman tilan, palauttaen ohjelman uuden tilan. Näin ohjelma on näennäisesti
puhtaan funktionaalinen ja sivuvaikutukseton. \cite{functionalreact}

Kuten tosielämän ohjelmille on tyypillistä, frontendissa tarvitaan myös sivuvaikutuksia joidenkin toiminnallisuuksien
toteuttamiseen. Sivuvaikutukset ovat hyödyllisiä esimerkiksi tiedon tallentamiseen selaimen muistiin ja satunnaisten
numeroiden tuottamiseen. Funktionaaliset frontend-ohjelmistokehykset mahdollistavatkin tyypillisesti toiminnallisuuden
sivuvaikutusten toteuttamiseen funktionaaliselle ohjelmoinnille tyypillisen rajatusti. \cite{elmlang}\cite{reactjs}

\section{Funktionaalinen reaktiivinen ohjelmointi}
Reaktiivinen ohjelmointi tarkoittaa asynkronisen datan ja tapahtumien kytkemistä muuttumattomiin datatyyppeihin
ohjelmointikielen tai ohjelmistokehyksen sisäänrakennetuilla tekniikoilla. Muuttumattomien datavirtojen käsittely
voidaan toteuttaa puhtaasti funktionaalisilla operaatioilla. Tämä on funktionaalisen reaktiivisen ohjelmoinnin (engl.
Functional Reactive Programming, FRP) keskeinen konsepti, jota esimerkiksi React ja Elm hyödyntävät
\cite{elmlang}\cite{reactjs}. \cite{fpmattered}

Funktionaalinen reaktiivinen ohjelmointi mahdollistaa perinteisten puhtaiden funktioiden (esim. map, filter, reduce)
käyttämisen sovelluksen kuvaamiseen. Keskeinen idea on, että ohjelman ympäristöstä tuleva data sidotaan näennäisesti
muuttumattomiin signaaleihin, joita voidaan käsitellä puhtaasti funktionaalisesti. Kun data muuttuu, suoritetaan
laskenta uudestaan uudella arvolla. Toteutus on tavallaan muuttumaton, vaikkakin data on muuttuvaa. Reaktiiviseen
ohjelmointimalliin soveltuvia tapahtumia ja datalähteitä frontend-web-sovelluksessa ovat muun muassa DOM-tapahtumat ja
rajapintakutsut. \cite{elmlang}