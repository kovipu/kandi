\chapter{Web-kehitys}

\section{World Wide Web}
World Wide Web tai WWW on järjestelmä tiedon jakamiseen, joka käyttää Internet-verkkoa. Web-sisältöä luetaan selaimella,
joka hakee sisällön web-palvelimelta HTTP-siirtoprotokollan avulla. Verkkosivujen sisällön kuvaamiseen on perinteiseti
ollut kolme tekniikkaa: HTML-kieli sivun sisällön kuvaamiseen, CSS-kieli sivun ulkoasun kuvaamiseen ja JavaScript-kieli
sivun toiminnallisuuden toteuttamiseen.\cite{javascriptguide}

\subsection{Staattiset verkkosivut}
Alkuperäinen tapa jakaa web-sisältöä on staattisena verkkosivuna (eng. static website). Staattinen verkkosivu koostuu
HTML-sivuista, dokumenteista ja mediasta, jotka luetaan suoraan web-palvelimen muistista, ilman että palvelin tekee
niihin muutoksia ajon aikana. Staattiset verkkosivut sisältävät usein HTML-, CSS- ja multimediasisältöä, sekä selaimessa
ajettavia skriptejä tai ohjelmia, jotka on kirjoitettu JavaScript-kielellä tai muulla selaimen tulkitsemalla kielellä.
\cite{staticdynamicwebsites}

% ehkä tutkimuskohteiden alle?
\subsubsection{Staattisen verkkosivun käännösvaihe}

\subsection{Dynaamiset verkkosivut}
Dynaaminen verkkosivu tarkoittaa sivua, joka ei ole suoraan selaimen luettavassa muodossa. Dynaamisen sivun web-
muodostaa selaimella renderöityvän sivun ajon aikana. Dynaamisen verkkosivun operaatioiden toteutus ei olekaan rajattu
selaimen tulkittaviin ohjelmointikieliin, vaan käytännössä minkä tahansa kielen käyttämninen on mahdollista. Yleisimpiä
kieliä dynaamisen verkkosivun toteuttamiseen ovat PHP, Python, Perl, Ruby, Java, C\# ja Node.js. Dynaaminen verkkosivu
tallettaa useimmiten sisältönsä tietokantaan.\cite{staticdynamicwebsites}

\section{Tutkimuskohteet}

\subsection{Dynaaminen sisällön hakeminen}

\subsection{Tapahtumankäsittely}
