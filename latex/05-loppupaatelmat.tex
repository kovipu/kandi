\chapter{Loppupäätelmät}
Funktionaalinen ohjelmointi on kasvattanut suosiotaan etenkin frontend-ohjelmoin\-nis\-sa viime vuosina.
Funktionaalisesta ohjelmoinnista onkin paljon hyötyjä verrattuna perinteiseen imperatiiviseen ohjelmointiin. Puhtaan
funktionaalisen ohjelman tilattomuus tekee ohjelmasta helpommin toistettavaa ja testattavampaa. Puhdas funktio on myös
automaattisesti säieturvallinen. Funktionaaliselle ohjelmointityylille on myös tyypillistä koodin kompaktiutta ja
luettavuutta parantavien abstraktioiden, kuten hahmonsovituksen, käyttö.

Funktionaalinen paradigma soveltuu erinomaisesti web-kehitykseen. Backend-ohjelmoinnissa suosittuja kieliä ovat muun
muassa Clojure- ja Elixir-kielet. Frontend-ohjelmoinnissa funktionaalinen reaktiivinen ohjelmointi on hyvin tyypillinen
toteutustapa verkkosovellukselle muun muassa React-kirjaston myötä.

Tässä tutkielmassa esitellyt teknologiat, React-kirjasto ja Elm-ohjelmointikieli ovat molemmat selaimen tulkittavaksi
JavaScript-koodiksi käännettäviä ohjelmistokehyksiä. Molemmat tekniikat toteuttavat funktionaalista
ohjelmointiparadigmaa, joskin Elm ei salli lainkaan muita paradigmoja. Näistä kahdesta etenkin React on saavuttanut
valtavan suosion, ja onkin kirjoitushetkellä maailman suosituin frontend-ohjelmistokehys. Sekä React että Elm
mahdollistavat funktionaalisen ohjelmoinnin piirteiden, kuten sivuvaikutusten välttämisen ja funktionaalisen
tilanhallinnan toteuttamisen.

Etenkin funktionaalinen reaktiivinen ohjelmointi on hyvin luonteva tapa toteuttaa frontend-verkkosovellus.
Tapahtumavirta on luonteva tapa kuvata frontendin tapahtumienkäsittelyä, ja se kytkeytyy helposti puhtaasti
funktionaalisiin tapahtumankäsittelijöihin. Valitettavasti tässä tutkielmassa käsiteltiin pääasiassa frontendin
toteutusta yleisesti funktionaalisella ohjelmoinnilla, ja reaktiivisen funktionaalisen ohjelmoinnin tutkimus jäi hyvin
pintapuoliseksi. FRP-ohjelmoinnin käyttäminen frontend-sovelluksen toteuttamiseen olisi luonteva jatko tälle
tutkielmalle.

Tämän tutkielman tutkimuskohteet on myös rajattu hyvin niukasti Reactin ja Elmin tiettyihin yksityiskohtiin. Molemmat
tekniikoista ovat funktionaalisen ohjelmoinnin kannalta erittäin käyttökelpoisia ja mielenkiintoisia, vaikka ne
tarjoavatkin erilaisia lähestymistapoja toteutuksiin. Molemmissa teknologioissa olisi kuitenkin vielä paljon
tutkittavaa. Frontend-verkkosovelluksen toteutukseen on myös useita muita mielenkiintoisia tekniikkavaihtoehtoja, muun
muassa puhtaasti funktionaaliset reaktiiviset RxJS- ja Bacon.js-kirjastot.
