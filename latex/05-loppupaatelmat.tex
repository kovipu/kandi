\chapter{Loppupäätelmät}
Funktionaalinen ohjelmointi on kasvattanut suosiotaan etenkin front-end-ohjelmoinnissa viime vuosina.
Funktionaalisesta ohjelmoinnista onkin paljon hyötyjä verrattuna perinteiseen imperatiiviseen ohjelmointiin. Puhtaan
funktionaalisen ohjelman tilattomuus tekee ohjelmasta helpommin järkeiltävää ja testattavampaa. Puhdas funktio on myös
automaattisesti säieturvallinen. Funktionaaliselle ohjelmointityylille on myös tyypillistä koodin kompaktiutta ja
luettavuutta parantavien abstraktioiden, kuten hahmonsovitus, käyttö.

Funktionaalinen paradigma soveltuu web-kehitykseen erinomaisesti, kuten monen funktionaalisen kielen suosiosta voi
päätellä. Back-end-ohjelmoinnissa suosittuja kieliä ovat muun muassa Clojure- ja Elixir-kielet. Front-end-ohjelmoinnissa
funktionaalinen ohjelmointityyli on merkittävästi suositumpaa, muun muassa React-kirjaston myötä. Funktionaalinen
responsiivinen ohjelmointi on hyvin tyypillinen tapa toteuttaa moderni front-end-verkkosovellus.

Tässä tutkielmassa esitellyt teknologiat, React-kirjasto ja Elm-ohjelmointikieli toteuttavat molemmat funktionaalista
ohjelmointiparadigmaa. Näistä etenkin React on saavuttanut valtavan suosion, ja onkin maailman suosituin front-end
ohjelmistokehys. Sekä React, että Elm mahdollistavat funktionaalisen ohjelmoinnin piirteiden, kuten sivuvaikutusten
ja funktionaalisen tilanhallinnan toteuttamisen.