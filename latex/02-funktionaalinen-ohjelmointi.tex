\chapter{Funktionaalinen ohjelmointi} \label{Funktionaalinen ohjelmointi}

\section{Määritelmä}
Funktionaalinen ohjelmointi on deklaratiivinen ohjelmointiparadigma, jossa ohjelman suoritus tapahtuu sisäkkäisten
matemaattisten funktioiden evaluoimisena. Dek\-laratiivisessa ohjelmoinnissa määritetään \textit{mitä} ohjelmoidaan, ei
\textit{miten}. Deklaratiivisessa ohjelmoinnissa ei ole implisiittistä globaalia tilaa, ja siten ohjelma koostuu
lausekkeista, joiden välillä tilaa siirretään eksplisiittisesti. Se on vastakohta perinteisemmälle imperatiiviselle
ohjelmoinnille. Funktionaalinen ohjelmakoodi pyrkii käsittelemään tilaa tarkemmin ja välttämään muuttuvia arvoja.
\cite{hudak}

Puhtaasti funktionaalinen ohjelmointikieli ei salli lainkaan muuttuvaa tilaa tai datan muokkaamista, eli niin kutsuttuja
sivuvaikutuksia (engl. side effect). Tämä vaatii sen, että funktioiden paluuarvo riippuu ainoastaan funktion saamista
parametreista, eikä mikään muu kuin parametrit voi vaikuttaa funktion paluuarvoon. Funktionaaliseksi ohjelmointikieleksi
lasketaan kuitenkin myös moniparadigmaiset ohjelmointikielet, jotka sisältävät funktionaalisia piirteitä.
Yksinkertaisten loogisten toimintojen lisäksi realistinen ohjelmointikieli vaatii epäpuhtauksia muun muassa siirrännän
(engl. Input/Output) toteuttamiseen. \cite{purelyFunctional} Moderneissa ohjelmointikielissä tähän on myös
funktionaalisia ratkaisuja, esimerkiksi Haskell-kielen IO-monadi \cite{learnhaskell}.

\section{Lambdakalkyyli}
Funktionaalinen ohjelmointi perustuu lambdakalkyyliin – formaalin laskennan malliin jonka kehitti Alonzo Church vuonna
1928. Lambdakalkyyliä pidetään ensimmäisenä funktionaalisena ohjelmointikielenä, vaikka se keksittiinkin ennen
ensimmäisiä varsinaisia tietokoneita, eikä sitä pidettykään aikanaan ohjelmointikielenä. Useita moderneja
ohjelmointikieliä, kuten Lispiä, pidetään abstraktioina lambdakalkyylin päälle. Lambdakalkyyli voidaan kirjoittaa myös
$ \lambda $-kalkyyli kreikkalaisella lambda-kirjaimella. \cite{hudak}

Churchin lambdakalkyyli koostuu vain kolmesta termistä: muuttujista, abstraktioista ja sovelluksista.
Muuttujat ovat yksinkertaisesti merkkejä tai merkkijonoja jotka kuvaavat jotain parametriä tai arvoa. Churchin
alkuperäinen lambdakalkyyli ei tuntenut muuttujien asettamista, ainoastaan arvojen syöttämisen parametrina. Abstraktio
$ \lambda x [ M ] $ määrittelee funktion parametrille x toteutuksella M. Sovellus $ \{ F \} ( X ) $ suorittaa funktion F
arvolla X. \cite{lambdacalculus}

\section{Konseptit}

\subsection{Tilattomuus}
Deklaratiivisissa ohjelmointikielissä ei ole implisiittistä tilaa, mikä eroaa imperatiivisista kielistä, joissa tilaa
voi muokata lauseilla (engl. statement) ohjelmakoodissa. Tilattomuus tekee laskennasta mahdollista ainoastaan
lausekkeilla (engl. expression). Funktionaaliset ohjelmointikielet käyttävät lambda-abstraktioita kaiken laskennan
perustana. Esimerkiksi logiikkaohjelmoinnissa vastaava rakenne on relaatio. Tilattomuus sallii vain muuttumattomat
vakiot, mikä estää muuttujien arvon muuttamisen ja perinteiset silmukat. Puhtaasti funktionaalisessa ohjelmoinnissa
rekursio on ainoa tapa toteuttaa toisto. Esimerkiksi funktio, joka laskee luvun \textit{n} kertoman voitaisiin toteuttaa
imperatiivisesti Python-kielessä esimerkiksi näin:
\begin{minted}{python}
def factorial(n):
    result = 1
    while n >= 1:
        result = result * n
        n = n - 1
    return result
\end{minted}
Funktionaalisessa ohjelmointikielessä toistorakenne on toteutettava rekursion avulla, mutta tästä on usein tehty
vaivatonta muun muassa yksinkertaisen syntaksin ja tehokasta väliaikaisten arvojen pois optimoinnin ansiosta.
\cite{hudak} Funktio, joka laskee luvun n kertoman voidaan toteuttaa esimerkiksi näin funktionaalisesti
Haskell-kielessä:
\begin{minted}{haskell}
factorial :: Int -> Int
factorial n = if n < 2
                then 1
                else n * factorial (n-1)
\end{minted}

\subsection{Korkeamman asteen funktiot}
Funktionaaliset ohjelmointikielet helpottavat funktionaalista ohjelmointityyliä muun muassa sallimalla korkeamman asteen
funktiot (engl. higher-order function).  Korkeamman asteen funktio eroaa tavallisesta funktiosta siten, että se voi
ottaa parametrinaan tai palauttaa funktion. Ensimmäisen luokan kansalainen on ohjelmointikielen rakenne, jonka voi
syöttää parametrina funktioille, asettaa paluuarvoksi funktiolle ja tallentaa tietorakenteisiin. Ohjelmointikielissä,
jotka tukevat korkeamman asteen funktioita, funktiot ovat ensimmäisen luokan kansalaisia. Korkeamman asteen funktioiden
avulla ohjelmakoodi ja data ovat jossain määrin vaihdettavissa, joten niiden avulla voidaan abstrahoida kompleksisia
rakenteita. \cite{hudak} Esimerkiksi map-funktio on korkeamman asteen funktio, joka suorittaa parametrina annetun
funktion listalle. Map-funktiota käytetään usein modernissa webin frontend-ohjelmoinnissa \cite{functionalreact}.
Esimerkiksi listaa voidaan käsitellä React-sovelluksessa map-funktiota käyttäen:
\begin{minted}{jsx}
items.map(item => <Item {...item} />);
\end{minted}

\subsection{Laiska laskenta}
Laiska laskenta (engl. lazy evaluation) on funktion laskentastrategia, jossa lausekkeen arvo lasketaan vasta kun sitä
ensimmäisen kerran tarvitaan, mutta ei aikaisemmin. Laiska laskenta on päinvastainen laskentastrategia innokkaaseen
laskentaan (engl. eager evaluation), jossa lausekkeen arvo lasketaan heti ensimmäisellä kerralla, kun lauseke
esitellään. Yleisimmät ohjelmointikielet käyttävät innokasta laskentastrategiaa. Laiska laskenta voi vähentää
suoritukseen kuluvaa aikaa ja siten voi parantaa ohjelman suorituskykyä eksponentiaalisesti esimerkiksi call by name
innokkaaseen laskentastrategiaan verrattuna, jossa lausekkeen arvo voidaan joutua laskemaan useita kertoja. Laiskan
laskennan toteuttavat useat puhtaasti funktionaaliset ohjelmointikielet, kuten Haskell. Myös jotkin moniparadigmaiset
kielet toteuttavat laiskan laskennan, esimerkiksi Scala-kieli mahdollistaa sen \textit{lazy val}-lausekkeen avulla.
\cite{languagedesign}

Laiska laskenta mahdollistaa päättymättömät tietorakenteet. Niin sanottujen laiskojen listojen loppupää evaluoidaan
vasta kun sitä käytetään laskennassa. Tämä näkyy useissa funktionaalisissa ohjelmointikielissä listan toteutuksena,
jossa listaa evaluoidaan alkupäästä loppua kohti. Esimerkiksi Haskellissa voidaan määrittää lista kaikista
kokonaisluvuista (Int-tietotyypin rajoissa) alkaen luvusta \textit{n}:
\begin{minted}{haskell}
from :: Int -> [Int]
from n = n : from (n+1)
\end{minted}
Funktion rekursiivinen kutsu aiheuttaisi innokkaasti laskevissa ohjelmointikielissä päättymättömän rekursion, mutta
Haskellin tapauksessa vain laskennan tarvitsema osa listasta evaluoidaan. \cite{languagedesign}

Funktionaalisissa ohjelmointikielissä häntärekursio ei aiheuta pinon ylivuotoa toisin kuin kielissä, joissa jokainen
kutsu varaa osan kutsupinosta. Näinollen päättymätön rekursio on toimiva vaihtoehto päättymättömälle silmukalle.
Esimerkiksi Haskell-kielen main-funktioita voidaan toistaa äärettömästi näin:
\begin{minted}{haskell}
main :: IO ()
main =
  do
    putStrLn "do something"
    main
\end{minted}

\subsection{Hahmonsovitus}
Hahmonsovitus (engl. pattern matching) on funktionaaliselle ohjelmoinnille tyypillinen piirre, jossa sama funktio
voidaan määritellä useita kertoja. Funktiomäärittelyistä vain yhtä sovelletaan tapauskohtaisesti purkamalla parametri
ja haarautumalla sen mukaan mikä sen "hahmo" on. Modernit funktionaaliset ohjelmointikielet, kuten Haskell,
mahdollistavat monimutkaisten tyyppien ja lausekkeiden dekonstruoimisen hahmonsovituksen avulla \cite{learnhaskell}.
Hahmonsovituksen toteutus on käytännössä case-lauseke, jossa lausekkeen ehto kuvaa syötetyn parametrin tietotyyppiä tai
sen rakenneta. Tämän avulla ohjelman suoritus jakautuu syötetyn muuttujan tyypin tai rakenteen mukaan. \cite{hudak}

Esimerkiksi kertomafunktio voidaan toteuttaa rekursiivisesti hahmonsovituksen avulla. Tässä esimerkissä rekursion
alkuarvo määritetään syötteelle \textit{0}, ja muut syötteet käsitellään parametrina \textit{n}.
\begin{minted}{haskell}
factorial :: Int -> Int
factorial 0 = 1
factorial n = n * factorial (n - 1)
\end{minted}

\section{Funktionaalisen laskentamallin seuraukset}
Puhtaasti funktionaalinen laskentamalli karsii ohjelmasta pois tietyt ohjelmointivirheiden tyypit. Ohjelman
monisäikeistys on hyödyllistä laitteiston tehokkaamman käytön kannalta, mutta vaatii useissa kielissä ohjelmoijalta
säikeiden synkronoimista ohjelmakoodissa. Puhdas sivuvaikutukseton laskenta on automaattisesti säieturvallista (engl.
thread safe) tarkoittaen, ettei sitä tarvitse manuaalisesti synkronoida. Koska puhtaan funktion lopputulos riippuu
ainoastaan syötteistä, ei syötteiden laskujärjestyksellä ole väliä. Laskennan lopputulosta voidaan siten pitää
deterministisenä, mikä helpottaa monisäikeistystä.

Toistettavuus (engl. reproducibility) tarkoittaa, että ohjelman tietynlainen toiminta on toistettavissa.
Deterministisyys ja toistettavuus parantavat ohjelman testattavuutta. Jos puhdas funktio toimii tietyllä tavalla kerran,
se toimii niin myös joka kerta ajon aikana. Toistettavuus helpottaa ajon aikana löydetyn virheellisen tilan toistamista
syöttämällä sama tarkkaan määritetty virheellinen tila testiympäristöön. \cite{functionaljava}

Funktionaaliset ohjelmointikielet ovat olleet hitaampia kuin niiden perinteisemmät imperatiiviset sukulaiset, kuten
C-kieli, mutta ero on ajan kuluessa kaventunut. Kääntäjäoptimoinneista huolimatta imperatiiviset ohjelmointikielet
ovat useimmiten optimoidumpia imperatiivisella laitteistolla. Imperatiivisille tietorakenteille on nykyään olemassa
monia funktionaalisia vaihtoehtoja, jotka voivat kääntyä optimaalisemmin kuin perinteiset funktionaaliset toteutukset.
\cite{functionaldatastructures}
