\keywords{funktionaalinen ohjelmointi, funktionaalinen reaktiivinen ohjelmointi, web-ohjelmointi, frontend, React, Elm}
\keywordstwo{functional programming, functional reactive programming, web-programming, frontend, React, Elm}

\begin{abstract}
Tietotekniikan käyttö arkielämässä on yleistynyt viime vuosina räjähdysmäisesti. Etenkin verkkosivujen ja -sovellusten
käyttö on suositumpaa kuin koskaan. Kasvanut käyttö on lisännyt merkittäväst web-tekniikoille asetettuja vaatimuksia
muun muassa luotettavuuden ja tuottavuuden kannalta. 

Funktionaalinen ohjelmointi on matemaattisten funktioiden evaluointiin perustuva ohjelmointiparadigma. Funktionaalinen
ohjelmointi ei ole paradigmana uusi keksintö, mutta sen soveltaminen web-kehitykseen on nostanut päätään vasta viime
vuosina. Tämän tutkielman tavoitteena on selvittää, miten funktionaalinen ohjelmointi soveltuu frontend-web-kehitykseen.

Tutkielmassa selvitetään ensin funktionaalisen ohjelmoinnin konsepteja ja etuja. Tämän jälkeen tarkastellaan
web-kehitystä, ja miten funktionaalinen ohjelmointi soveltuu frontend-web-sovelluksen toteutukseen. Tutkielmassa
tarkastellaan tarkemmin kahta funktionaalista reaktiivista frontend-ohjelmointitekniikkaa: React-kirjastoa ja
Elm-ohjelmointikieltä.

Funktionaalinen ohjelmointi ja etenkin funktionaalinen reaktiivinen ohjelmointi on hyvin luonteva tapa toteuttaa
frontend web-sovellus. Tutkituista tekniikoista molemmat tukevat funktionaalista ohjelmointityyliä, mutta Elm ei
myöskään salli muuta tapaa toteuttaa ohjelma.
\end{abstract}
