\chapter{Functional programming} \label{Functional programming}

% Functional programming paradigm
% - Based on lambda calculus
% - A bit of history

% - The paradigm
Functional programming is a computer programming paradigm in which the evaluation of a computer program is carried out
through the evaluation of deeply nested mathematical functions. Declarative programming is a programming paradigm that
describes the program only as expressions or declarations instead of statements. Functional programming is a declarative
programming paradigm that describes the program as function declarations. Functional code avoids changing-state and
mutable data, which can make understanding a program easier.

Functional programming is largely based on lambda calculus, a formal system developed in the 1930s by Alonzo Church.
Many functional programming languages, such as Lisp, can be seen as abstractions on top of lambda calculus. \cite{hudak}
% - modern lambda calculus

\section{Concepts}
% * declarative
% * immutability
% * higher order functions
% * lazy evaluation

\section{Languages}
% - Lisp vs ML
% - Modern functional languages. Clojure, Haskell etc

