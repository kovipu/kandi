\chapter{Funktionaaliset front-end ohjelmistokehykset}
Tämän tutkielman tutkimuskohteiksi on valittu funktionaaliset reaktiiviset ohjelmistokehykset React-kirjasto ja
Elm-ohjelmointikieli. React-kirjastoa ja Elm-kieltä tutkitaan yleisesti funktionaalisen ohjelmoinnin kannalta, sekä
tarkemmin siitä, miten ne toteuttavat tilankäsittelyn, mahdollistavatko ne sivuvaikutukset ja miten toteuttavat
asynkroniset tapahtumat.

\section{Ohjelmistokehysten esittely}
React on Facebookin ylläpitämä avoimen lähdekoodin käyttöliittymäkirjasto, jota käytetään tyypillisesti
JavaScript-kielen kanssa. React on alunperin julkaistu vuonna 2013 ja se on kirjoitushetkellä vuonna 2020 maailman
suosituin front-end ohjelmistokehys \cite{npmtrends}. React-kirjaston syntaksi on deklaratiivista, ja se mahdollistaa
käyttöliittymän pilkkomisen komponentteihin. React ei ota kantaa muihin tekniikkaratkaisuihin, vaan keskittyy pelkästään
yksittäisen DOM-puun hallitsemiseen. Ohjelmistokehittäjälle jää näin täysi valta valita muut teknologiat vapaasti.
\cite{reactjs}

Elm on funktionaalinen ohjelmointikieli, joka käännetään JavaScript-kieleksi. Elmin on alunperin kehittänyt Evan
Czaplicki osana maisterin tutkielmaansa (eng. Master's thesis) vuonna 2012. Elm ei ole tutkielman kirjoitushetkellä
saavuttanut suurta suosiota, vaan sen NPM-paketinhallinnan latausmäärä on vain murto-osa React-kirjaston latausmäärästä 
\cite{npmtrends}. Elm keskittyy web-pohjaisten graafisten käyttöliittymien deklaratiiviseen toteuttamiseen. Elm
kytkeytyy tavallisen HTML DOM-puun solmuun. Elm ei myöskään ota kantaa muihin tekniikkaratkaisuihin, mutta sen kanssa on
mahdotonta käyttää muilla kielillä kirjoitettuja komponentteja ja kirjastoja. \cite{elmlang}

\section{Funktionaalinen ohjelmointi}

// miten mahdollistavat funktionaalista ohjelmointia

\section{Funktionaalinen tilankäsittely}

// Reactissa: useState, useReducer, Redux

// Elmissä: Model-abstraktio

\section{Sivuvaikutukset}

// Reactissa: imperatiivinen ohjelmointi, useEffect

// Elmissä: Elm-runtime abstraktoi, innokas laskenta aiheuttaa sivuvaikutuksia

\section{Funktionaalinen tapahtumankäsittely}

// eventit, api-kutsut \& asynkronisuus
