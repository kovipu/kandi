\chapter{Funktionaaliset front-end ohjelmistokehykset}
Tämän tutkielman tutkimuskohteiksi on valittu funktionaaliset reaktiiviset ohjelmistokehykset React-kirjasto ja
Elm-ohjelmointikieli. React-kirjastoa ja Elm-kieltä tutkitaan yleisesti funktionaalisen ohjelmoinnin kannalta, sekä
tarkemmin siitä, miten ne toteuttavat tilankäsittelyn, mahdollistavatko ne sivuvaikutukset ja miten toteuttavat
asynkroniset tapahtumat.

\section{Ohjelmistokehysten esittely}
React on Facebookin ylläpitämä avoimen lähdekoodin käyttöliittymäkirjasto, jota käytetään tyypillisesti
JavaScript-kielen kanssa. React on alunperin julkaistu vuonna 2013 ja se on kirjoitushetkellä vuonna 2020 maailman
suosituin front-end ohjelmistokehys \cite{npmtrends}. React-kirjaston syntaksi on deklaratiivista, ja se mahdollistaa
käyttöliittymän pilkkomisen komponentteihin. React ei ota kantaa muihin tekniikkaratkaisuihin, vaan keskittyy pelkästään
yksittäisen DOM-puun hallitsemiseen. Ohjelmistokehittäjälle jää näin täysi valta valita muut teknologiat vapaasti.
\cite{reactjs}

Elm on funktionaalinen ohjelmointikieli, joka käännetään JavaScript-kieleksi. Elmin on alunperin kehittänyt Evan
Czaplicki osana maisterin tutkielmaansa (eng. Master's thesis) vuonna 2012. Elm ei ole tutkielman kirjoitushetkellä
saavuttanut suurta suosiota, vaan sen NPM-paketinhallinnan latausmäärä on vain murto-osa React-kirjaston latausmäärästä 
\cite{npmtrends}. Elm keskittyy web-pohjaisten graafisten käyttöliittymien deklaratiiviseen toteuttamiseen. Elm
kytkeytyy tavallisen HTML DOM-puun solmuun. Elm ei myöskään ota kantaa muihin tekniikkaratkaisuihin, kuten tyylikieleen,
mutta Elm-koodissa on mahdotonta käyttää muilla kielillä kirjoitettuja komponentteja ja kirjastoja. \cite{elmlang}

\section{Funktionaalinen ohjelmointi}
React korostaa funktionaalista ohjelmointia yli imperatiivisen ohjelmoinnin, mutta sallii silti kaiken tavallisen
imperatiivisen JavaScript-syntaksin. React ei pystykään välttämään muuttuvia arvoja tai sivuvaikutuksia, mutta rohkaisee
silti ohjelmoimaan funktionaalisesti, muun muassa tarjoamalla tilanhallinan, joka soveltuu funktionaaliseen
ohjelmointiin. \cite{reactjs}

Elm pyrkii olemaan puhtaasti funktionaalinen ohjelmointikieli, eikä salli lainkaan esimerkiksi muuttuvia arvoja tai
epäpuhtaita funktioita. Elm ei myöskään salli tilanteita, jotka voisivat aiheuttaa ajonaikaisia virheitä, vaan pakottaa
virheiden käsittelyn funktionaalisesti datana. Elm-kielen ajonaikanen ohjelmistokehys abstraktoi näennäisesti kaikki
epäpuhtaudet sisällensä, joten Elm-kielellä toteutettu ohjelma on näennäisesti puhtaan funktionaalinen. \cite{elmlang}

\section{Funktionaalinen tilankäsittely}
React tukee lokaalia tilaa missä tahansa komponentissa, joka on ristiriidassa funktionaalisen paradigman kanssa. Tila
React-kirjastossa on kuitenkin muuttumaton datarakenne, jonka muuttaminen vaatii sen korvaamisen uudella datalla. Tilan
muokkaamiseen on perinteisesti käytetty React-kirjaston Component-luokan setState-metodia, joka ylikirjoittaa tilan.
% tähän koodiesimerkkiin tarvitsee kaiketi viitata jotenkin.
\begin{minted}{js}
// väärin
this.state.greeting = 'Moikku';
// oikein
this.setState({ greeting: 'Moikku' });
\end{minted}
React-kirjasto tukee nykyään vakiona myös epäpuhtauksia funktioiksi abstraktoivia koukkuja (eng. hooks). Koukut
mahdollistavat esimerkiksi tilankäsittelyn puhtaassa funktionaalisessa komponentissa, ilman että tarvitaan luokkaa.
% tähän tarttis varmaan kans viitata.
\begin{minted}{jsx}
const Example = () => {
  // greeting on tila, joka muuttuu reaktiivisesti
  // setGreeting on funktio, jonka kutsuminen muuttaa tilan
  const [greeting, setGreeting] = useState('Moikku');
  return (
    <h1>{greeting}</h1>
  );
}
\end{minted}
Moderni React tukee myös koko sovelluksen globaalia tilanhallintaa, jota muokataan puhtailla funktiokutsuilla.
Tämänkaltaisen tilanhallinan toteuttamiseen voi käyttää React-kirjaston useReducer-koukkua. React-kirjaston kanssa
voidaan myös käyttää kolmannen osapuolen tilanhallintakirjastoa, esimerkiksi Redux on suosittu esimerkki
\cite{functionalwebdev}. \cite{reactjs}
% koodiesimerkki reducerista?

// Elmissä: Model-abstraktio

\section{Sivuvaikutukset}

// Reactissa: imperatiivinen ohjelmointi, useEffect

// Elmissä: Elm-runtime abstraktoi, innokas laskenta aiheuttaa sivuvaikutuksia

\section{Funktionaalinen tapahtumankäsittely}

// eventit, api-kutsut \& asynkronisuus
