\documentclass[12pt, titlepage]{article}
\usepackage[a4paper,bindingoffset=0.2in,%
            left=1in,right=1in,top=1in,bottom=1in,%
            footskip=.25in]{geometry}
\usepackage{amssymb,amsthm,amsmath}
\usepackage[english]{babel}

\usepackage{graphicx} %the document includes pictures

\linespread{1.24} %line spacing 1.5
\sloppy

\title{Functional programming in front-end web development \\ {\normalsize
B.Sc. (Tech.) thesis, Software Engineering}}

\author{Konsta Purtsi}
\date{June 2019}

\makeatletter
\let\inserttitle\@title
\let\insertworktype\@worktype
\makeatother

\makeatletter
\let\insertauthor\@author
\makeatother

\makeatletter
\let\insertdate\@date
\makeatother

\begin{document}

\pagestyle{empty}

\vspace*{\fill}
\begin{center}\Large %
\inserttitle
\end{center}

\vspace{0.3cm}
\begin{center}
\insertauthor \\
\insertdate
\end{center}

\vfill
\begin{center}
University of Turku\\
Department of Future Technologies
\end{center}

\begin{center} %TODO: Translate to English
\noindent {\footnotesize Turun yliopiston laatujärjestelmän mukaisesti tämän julkaisun alkuperäisyys on tarkastettu Turnitin OriginalityCheck -järjestelmällä.}
\end{center}

\newpage
\tableofcontents

\newpage

\pagestyle{plain}
\setcounter{page}{1}

\section{Introduction}
% Max 2-3 pages

\subsection{Thesis motivations}
% Why is this a thing / motivation speech

\subsection{Research question}
% Overview on functional programming
% Overview on web development

\subsection{Work methodology}
% Work methodology

\subsection{Thesis structure}
% Structure of the paper

\section{Functional Programming}
% Functional programming paradigm
% - Based on lambda calculus
% - A bit of history
% - Languages
%   - Lisp vs ML

% - The paradigm
Functional programming is a computer programming paradigm in which the evaluation of a computer program is carried out
through the evaluation of deeply nested mathematical functions. Declarative programming is a programming paradigm that
describes the program only as expressions or declarations instead of statements. Functional programming is a declarative
programming paradigm that describes the program as function declarations. Functional code avoids changing-state and
mutable data, which can make understanding a program easier.

Functional programming is largely based on lambda calculus, a formal system developed in the 1930s by Alonzo Church.
Many functional programming languages, such as Lisp, can be seen as abstractions on top of lambda calculus. \cite{hudak}
% - modern lambda calculus

\subsection{Concepts}
% * declarative
% * immutability
% * higher order functions
% * lazy evaluation

\section{Web Development}

\subsection{Definition}

\subsection{Subjects of study}
% Dynamic data fetching \& Event handling
% Research question

\section{Studied technologies}
\subsection{JavaScript \& React}
\subsection{Elm}
\subsection{ClojureScript \& Re-frame}

\section{Example applications}
\subsection{JavaScript \& React}
% Juomamaatti drink ordering system

\subsection{Elm}
% 2048 online game

\subsection{ClojureScript \& Re-frame}
% TBD

\subsection{Findings}
% Findings of success with different technologies go here.

\section{Conclusion}
% Nothing new is introduced here, only conclusions of the findings. "There's more research to do"

\addcontentsline{toc}{section}{References}
\renewcommand{\refname}{References}

\begin{thebibliography}{9}

\bibitem{hudak}
Paul Hudak.
\textit{Conception, Evolution, and Application of Functional Programming Languages}

\end{thebibliography}

\end{document}
